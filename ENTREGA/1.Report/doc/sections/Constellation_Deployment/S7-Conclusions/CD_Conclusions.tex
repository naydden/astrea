\section{Conclusions}
This final section is intended to put and end to the Constellation Deployment Department activities. First of all, a brief summary of the work done is carried out, secondly, the compliance of the tasks assigned to this department in the Project Charter document is verified. 
Accomplished tasks: 
\begin{itemize}
\item Launching System: a launching platform has been chosen regarding all the important parameters. Electron, from the enterprise Rocket Lab is the rocket that will bring Astrea Constellation to life. 
\item Deployer: a suitable deployer has been selected according to the standards of CubeSat deployment. GPOD deployer, developed by the enterprise GAUSS is in charge of the separation of the CubeSats from the rocket. 
\item First Placement: the assembly of the satellites will begin approximately 420 days before the first launching. The first placement will consist on eight launchings (one per orbital plane) and will last eight weeks. 
\item Replacement Strategy: similar to the first placement strategy, new orbital planes are placed between the old ones avoiding the formation of gaps during the decay of the satellites that are being renewed.
\item Spare Strategy:
\item End of Life Strategy: an uncontrolled de-orbit procedure has been chosen. 
\end{itemize}
The summary shown above demonstrates that the Constellation Deployment Department has fulfilled the requested duties. According to the legislation, both the chosen launcher (Electron) and deployer (GPOD) are certified. Their enterprises designed them strictly following the international requirements. 

