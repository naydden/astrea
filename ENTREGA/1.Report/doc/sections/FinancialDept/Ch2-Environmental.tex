\chapter{Environmental Impact Study}

\section{Introduction}
This chapter pretends to assess the environmental consequences (positive and negative) of developing the project. The target of this study is to identify, predict, evaluate and mitigate the biophysical and social negative effects that the project could generate during the execution of it.


\section{Ground Stations}

At first sight the Ground Stations do not represent any environmental problem. The main factor that has to be taken into account is the placement of the stations. They have to be located in a place where they do not interfere with the ecosystem. The placement of the stations has to be adequate with the environmental legislation of the countries. 

\section{Satellites}

For analysing the impact of the satellites it has to be studied the possible environmental impact during the fabrication and during the orbital life.

Since the fabrication of the satellites is externalized to other companies, the responsibility of the environmental consequences derived of this manufacturing is over these companies. For commercializing these products they must pass all the controls required.

During the orbital performance of the satellites, it has to be taken into account whether or not they would become orbital waste. The satellites are designed to burn out in the atmosphere at the end of their useful life. This burnt should not leave any solid residue that could precipice over the surface. The deorbit would be forced and controlled by the propulsion system of the satellite. In the case that this system fails, given that they will orbit in a LEO, they will be deorbited and burnt out naturally in a period around 5 years.

\section{Launch system}
The most critical part of the entire process, in environmental terms, is the launch of the satellites. For this reason the main relevance in this report is given to the spacecraft that will put the satellites in orbit, the Electron rocket of Rocket-Lab.

The company operate in New Zeeland, and for doing it, the Ministry for the Environment make an accurate study of the environmental impact of the Electron launching. The entire document can be seen at \cite{EIS}.

In this document are analysed the critical components of the spacecraft:
\begin{itemize}
\item \textbf{Structure}. The primary structural material is carbon fibre reinforced polymer. The carbon filaments are chemically inert and do not react to seawater.
\item \textbf{Propellants}. Liquid oxygen and kerosene (RP-1 analogue) propellants are used on both the first and second stages of the launch vehicle. Liquid oxygen, if released to the atmosphere, rapidly boils and returns to the atmosphere as gaseous oxygen. RP-1 kerosene is a highly refined grade of hydrocarbon with low density, a thin surface film and rapid evaporation.
\item \textbf{Pneumatics}.All inflight pneumatic systems use stored pressurised cold gases to provide tank pressurisation, cold-gas manoeuvring thrust in space, and for stage separation mechanisms. All gases are non-toxic.
\item \textbf{Engines}. The launch vehicle uses nine engines for stage 1 and a single engine for stage 2. The engines are constructed of inconel, an inert high performance, corrosion resistant nickel alloy. At stage 1 separation, the thrust section is likely to separate from the stage, return to Earth’s surface and land in the Exclusive Economical Zone.
\item \textbf{Batteries}. The first stage batteries are highly likely to burn-up before returning to Earth’s surface. The stage 2 batteries will entirely burn-up downrange, with only the first battery potentially landing in the EEZ. The batteries are lithium-based, and contain no lead, acid, mercury, cadmium, or other toxic heavy metals.
\end{itemize}

The document also evaluates the following possible risks:
\begin{itemize}
\item \textbf{Risk of toxic effects}. The toxic effects of the materials comprising stage 1, the fairings and the two stage 2 LithiumIon batteries were assessed as low at all levels of launch activity.
\item \textbf{Risk of ingestion of materials and provision of floating shelter}. Floating jettisoned materials as shelter for pelagic organisms and the ingestion of jettisoned materials were both evaluated as having low ecological risk at all levels of launch activity.
\item \textbf{Environmental effect of the displacement of fishing activities.} For the demersal fish
and mobile invertebrate community, marine mammals and seabirds, the effects of fishing displacement would be low because these populations could also be impacted in the areas to which fishing is displaced. In the eastern jettison zone there is less fishing activity so the consequences of fishing displacement on the seabed community, demersal fish and mobile invertebrates, marine mammals and seabirds are negligible, reaching minor impacts after 1000 or more launches.
\item \textbf{Effect of the provision of hard substrates.} Another potential positive outcome for seafloor biota requiring hard substrates is that the jettisoned materials would provide further attachment sites. However, even after 10,000 launches this would provide only about 50 ha of additional attachment surface, leading to a moderate benefit at most.
\item \textbf{Disturbance to marine fauna.} Noise and disturbance to marine fauna above and below water is a potential consequence of
the jettisoned materials falling into the jettison zone. The chance of repeated disturbance to the same individuals or groups of marine mammals or seabirds increases with the number of launches. This was assessed as a low risk for up to 100 launches over two years, a moderate risk for up to 1000 launches over almost 20 years, and a high risk for up to 10,000 launches over almost 200 years.
\item \textbf{Risk of direct strikes causing mortality to components of the ecosystem.} Direct strikes causing mortality are a low risk for all components of the ecosystem up to 1000 launches over an almost 20 year period. Direct strikes reach moderate levels of risk for the benthic invertebrate community, sensitive benthic environments, and a rare threatened species, the magenta petrel, after 10,000 launches over a period of almost 200 years.
\item \textbf{Risk of smothering of sea floor organisms.} Smothering the feeding or respiratory structures of sea floor organisms by jettisoned materials was assessed as a low risk for all levels of launches up to 1000 launches and a moderate risk by 10,000 launches. This is likely to be a factor principally in areas of hard substrate where the jettisoned materials are unlikely to become buried in sediment so will be important principally on the Bounty Platform.
\end{itemize}
New Zealand legislation does not yet regulate these activities, since Rocket Lab is the first company that pretends to operate rocket launchings in the territory. The study concludes that the environmental effects of the activity may become significant after 10,000 launches, this would take 200 years to reach at one launch per week. The regulatory regime would have been reviewed well before this number of launches. During this review the Ministry allows the activity of the company.



