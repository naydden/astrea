\section{Thruster}
 
Thruster is a main part of the structure because it is needed to allow the satellite to realise different maneuvers how incorporate it adequatly to the orbit after the deployment of the rocket, can obtain the optimal orientation or to mantain the satellite in the orbital and avoid its fallen. 

The main parameters that must consider are thrust, total specific impulse,power required, weight of the  propulsion subsystem and its volume.

\paragraph{}
At the moment, the most used and more modern thrusters for satellites are: ionic, pulsed plasma, electrothermal and green monoprop thrusters. An important aspect to consider is that we are interested in is reducing the mass required although this will cause minor accelerations than conventional engines but it will be suitable for small satellites.

\paragraph{}
After a market study, an ionic thruster has been elected how the best option. The causes of this election are that the volume of the all propulsion subsystem and its weight are very small, specifict impulse is very high,thrust is acceptable and power requiered can be supplied by the solar panels.

The following table shows the main parameters of this thruster.

\begin{longtable}{| l | r |}

\hline
\rowcolor[gray]{0.60}	\textbf{BIT-1 ION THRUSTER} \\
\hline

\hline
\rowcolor[gray]{0.75}	\textbf{PARAMETERS} &  \textbf{VALUE}   \\
\hline

\cellcolor[gray]{0.85} \textbf{Total thruster power} & 10 W  \\
\cellcolor[gray]{0.85} \textbf{Thrust} & 100 uN \\
\cellcolor[gray]{0.85} \textbf{Specific impulse} & 2150 s \\
\cellcolor[gray]{0.85} \textbf{Thruster Mass} & 53 g \\
\cellcolor[gray]{0.85} \textbf{Propellant mass flow} & 4.9 ug/s \\
\cellcolor[gray]{0.85} \textbf{Grid input voltage} & 2 kV \\
\cellcolor[gray]{0.85} \textbf{Ion beam current} & 1.5 mA \\
\cellcolor[gray]{0.85} \textbf{Propellant utilization} & 41 percent \\
\cellcolor[gray]{0.85} \textbf{Energy Efficiency} & 27 percent \\
\hline

\end{longtable}


