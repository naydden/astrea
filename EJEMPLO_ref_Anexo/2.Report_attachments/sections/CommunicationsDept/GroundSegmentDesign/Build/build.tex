To start with what will involve building a GS, the parts of it will be exposed now:
\begin{itemize}
\item \textbf{Antenna}: For Astrea constellation a S and X band antennas will be needed in order to be able to communicate with the other nodes of the constellation.
\item \textbf{Transciever}: This part is the responsibl of receiving the signal from the antennas or emitting it to them. Depending on its kind, it can interpret or generate (respectively) that signal, or it can just be seen as a ADC. \item \textbf{Rotors system for pointing the antenna}:This system should be feeded with the satellite position and they have to point the antenna towards it. Therefore, a link between the received signal going trough to the computer and then back to the rotors should be established. 
\item \textbf{Computer for signal generating and interpreting}
\end{itemize}

The approximated cost of a GS is shown in the following table\footnote{The data has been extracted from\cite{XBand}\cite{SBand} }

\begin{table}[H]
\begin{center}
\begin{tabular}{|c|c|}
\hline
\textbf{Concept}&\textbf{Cost(\euro)}\\
\hline
X-band system&100000\\
\hline
X-band maintenance&20000\\
\hline
S-band system&46500\\
\hline
S-band maintenance&10000\\
\hline
Building&44440\\
\hline
\end{tabular}
\caption[Aproximation of costs of a Ground Station]{First approximation of costs of the Ground Station}
\end{center}
\end{table}

Then, the total initial investment for a ground station will be of approximately 190940\euro, and the annual cost of 30000\euro.