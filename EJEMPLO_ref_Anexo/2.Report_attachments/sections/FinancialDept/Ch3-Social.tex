\chapter{Legislation}
The legislation concerning activities related to space is the Space Law. Space Law is an international law comprised of international treaties and agreements. Its most important rules are the five international treaties, which have been developed under the supervision of the United Nations. The body that promotes these regulations is the United Nations Office for Outer Space Affairs (UNOOSA).
\newline
\newline
The international law is only applicable to the states that are parties to the treaties. According to the Outer Space Treaty, states are responsible for their national space activities, public or private. For this reason, each state usually adopts its national space regulations.
\newline
\newline
Another requirement of the UNOOSA is that space objects have to be registered, so all the member states have to establish their own national registries and provide this information to the United Nations Register. UNOOSA updates all the registrations through its website and through the United Nations Official Document System so that they are publicly available to anyone.
For example, in the United States of America a license is required to operate a space system. This license is given by the National Oceanic and Atmospheric Administration (NOAA). Its register is the U.S. Registry of Objects Launched into Outer Space. It also has an agency that identifies and tracks satellites, the Joint Space Operations Center, which provides some recommendations on how to develop, launch and operate a satellite. These are the regulatory requirements that the CubeSat developers recommend to fulfill because they are based in California, United States.
\newline
\newline
However, in the case of the Astrea constellation, since the company is based in Spain (a party of the Space Law), the current legislation is the \textit{Real Decreto 278/1995} of 24 February 1995. According to this Royal Decree \cite{Espana.MinisteriodelaPresidencia1995}, the objects launched from Spain or whose launch has been promoted by Spain, should be registered in the \textit{Registro Español de Objetos Espaciales Lanzados al Espacio Ultraterrestre} (Spanish Registry of Objects Launched into Outer Space). The necessary data to register the satellite must be provided to the \textit{Dirección General de Tecnología Industrial del Ministerio de Industria y Energía} (Department of Industrial Technology of the Ministry of Industry and Energy). This department will notificate the registry to the Secretary-General of the United Nations.
\newline
\newline
The registration has to contain the following data:
\begin{enumerate}[label=\alph*)]
\item Name of launching State or States;
\item An appropriate designator of the space object or its registration number;
\item Date and territory or location of launch;
\item Basic orbital parameters, including:
\begin{enumerate}[label=\Roman*)]
\item Nodal period;
\item Inclination;
\item Apogee;
\item Perigee;
\end{enumerate}
\item General function of the space object.
\end{enumerate}
and any other useful information.
For example, in the case of one of the Astrea satellites, the registration will be:
\begin{enumerate}[label=\alph*)]
\item Name of launching State or States: Spain
\item Designator of the space object or its registration number: AstreaSAT 1
\item Date and territory or location of launch: 22 February 2018, New Zealand
\item Basic orbital parameters: Low Earth Orbit
\begin{enumerate}[label=\Roman*)]
\item Nodal period: 95,4815 minutes
\item Inclination: 72 degrees
\item Apogee: 6.913,0 km
\item Perigee: 6.913,0 km
\end{enumerate}
\item General function of the space object: CubeSat 3U, part of the communications constellation Astrea
\end{enumerate}

Usually the launching state and the state of the launch are different. In this case the states have to determine which of them will register the launching object. It is common that the object is registered in the register of the launching state, the state that promotes the launch.