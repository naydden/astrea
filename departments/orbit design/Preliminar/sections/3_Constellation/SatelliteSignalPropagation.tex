\section{Satellite to Satellite Communications}


\subsection{Satellite signal propagation}

\textbf{Introduction}\newline

Satellite communication and propagated satellite-to-satellite signals are affected by the environmental conditions in which these operate. Some of these atmospheric effects will be important depending on the altitude of the satellites. \newline

\textbf{Ionospheric Effects}\newline

The ionosphere region is located between 60 and 1000 km of height. This atmospheric layer is especially affected by the extreme ultra-violet and X-ray radiations coming from the Sun, thus removing outer electrons from the atoms and creating an ionised-gas based region that will reflect incident radio signals. These conditions cause a variety of effects on propagating electromagnetic signals which must be taken into account. \newline

Refraction (bending) and dispersion are important issues to be considered. The first will cause the radars of the satellites to see the other satellite displaced from where it really is. The higher the frequency of the radio, the less the bending. Dispersion, on the other hand, will cause signal delay in wideband communication systems. \newline

\textbf{Scintillation}\newline

The ionospheric layer contains irregularities regards the density of the gas, and scintillation must be faced. Scintillations manifest themselves as a combination of variations of amplitude, phase, and polarisation angle. Signal angles of arrival can also be changed.These are more intense in equatorial regions, falling with increasing latitude away from the equator but then rising at high latitudes where auroras take place. The effects are also found to decrease with increasing frequency, and generally not noticeable above frequencies of 1 - 2 GHz.\newline

\textbf{Faraday Rotation}\newline

The combination of plasma in the ionosphere and the Earth's magnetic field can induce a rotation of the polarised radio signal which is travelling through this atmospheric layer. A rotation of 90 degrees may cause a total loss of signal. At high frequencies, i.e 4 GHz, the rotation will rarely exceed a few degrees.