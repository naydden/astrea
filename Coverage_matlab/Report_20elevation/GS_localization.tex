\documentclass[12pt,a4paper]{report}
\usepackage[utf8]{inputenc}
\usepackage[english]{babel}
\usepackage{amsmath}
\usepackage{amsfonts}
\usepackage{amssymb}
\usepackage{graphicx}
\usepackage{eurosym}
\usepackage[left=2cm,right=2cm,top=2cm,bottom=2cm]{geometry}
\usepackage{wrapfig}
\usepackage{mathdots}
\usepackage{caption}
\usepackage{cite}
\usepackage{mathrsfs}
\usepackage{float}
\author{Communications department}
\title{Ground Station localization}

\begin{document}
\maketitle
\tableofcontents
\listoffigures

\chapter*{Aim of the document}
\paragraph{}
Until now, it has been talked about the Ground Stations independently of wherever they are. It has to be studied where to place the Ground Stations in order to optimize the Ground Segment performance. This decision will depend mainly of the constellation characteristics, the earth topography and the country legislation and resources. 
\paragraph{}
This report will collect the analysis and procedures for arriving to the final decision of where the Ground Stations would be placed.

\chapter{Coverage analysis}
\section{Aim} 
\paragraph{}
Given the constellation topology, it will be studied the coverage of a Ground Station depending on its longitude and latitude. The aim of this analysis is to show where a Ground Station would have more coverage and propose a first approximation and proposal of the 3 Ground Station placement.

\section{Method}
For this porpoise it is developed a Matlab algorithm which which calculates, in a given moment, how many satellites can see an make contact a Ground Station in a given place of the earth. It has to be said that this analysis must be done during time since the satellites and the Earth are constantly moving. For doing that the algorithm follows the steps below:
\begin{enumerate}
\item Calculate where the satellites are refereed to an inertial Cartesian coordinates system, with the origin at the center of the Earth. This state analysis its done for a several time period with an adequate time-step
\item Calculate the Ground Station position refereed to the mentioned system. Since the system is inertial, the Ground Station will describe a circle in the rotational plane of the Earth relative to this system. This trajectory depend on the latitude and longitude of the place. This position is calculated for the same time period.
\item Calculate, for each time step, how many links can the station make. It will depend on the angle between the station and every satellite, and also the minimum elevation angle.
\end{enumerate}
\paragraph{}
Once this algorithm is tested and verified, it is studied the links during the day for several longitudes and latitudes and how this parameters affect to the coverage of the station.

\section{Latitude analysis}
\paragraph{}
It is easy to see that the latitude changing effect is practically independent for the longitude. For this reason, it is studied the links during the day for a given longitude. 
\paragraph{}
Doing the analysis for latitudes between 0º and 90º during 2 days, with a 5 minutes time-step, this are the results:

\begin{figure}[H]
\begin{center}
\includegraphics[scale=0.30]{0_30_90_lat.jpg}
\caption{Links vs time for latitudes from 0º to 90º}
\end{center}
\end{figure}

\paragraph{}
At it is shown in the figure 1.1 the behaviour is not constant during the day. For every day there is a peak and a valley. This is produced for the cylindrical asymmetry of the constellation. It has to be ensured that there is at least 1 link at all times. 0º and 30º are unacceptable latitudes. It is seen also that at the north pole it not any coverage at all.
\paragraph{}
Doing the same for negative latitudes the following results are obtained:

\begin{figure}[H]
\begin{center}
\includegraphics[scale=0.30]{0_-30_-90_lat.jpg}
\caption{Links vs time for latitudes from 0º to -90º}
\end{center}
\end{figure}

\paragraph{}
Comparing the results of the figure 1.2 with the figure 1.1 it is seen that they are practically the same with an offset of 12 hours. They are also seen small local deviations, but these are not much significant because of the time-step. This time-step is of 5 minutes for a first sight of the tendencies, and it do not allow extremely precise results.
\paragraph{}
Taking in account that the results of positive latitudes can be extrapolated to negative ones, the rest of the analysis is done for positive latitudes. At this point, the non all day coverage latitudes must be discarded and make a more accurate study around 60º:

\begin{figure}[H]
\begin{center}
\includegraphics[scale=0.30]{40_10_80_lat.jpg}
\caption{Links vs time for latitudes from 40º to 80º}
\end{center}
\end{figure}

\paragraph{}
Following the same logic:

\begin{figure}[H]
\begin{center}
\includegraphics[scale=0.30]{45_5_65_lat.jpg}
\caption{Links vs time for latitudes from 45º to 65º}
\end{center}
\end{figure}

\begin{figure}[H]
\begin{center}
\includegraphics[scale=0.30]{50_2,5_60_lat.jpg}
\caption{Links vs time for latitudes from 50º to 60º}
\end{center}
\end{figure}

\paragraph{}
The figure 1.5 suggest that between 50º and 60º there is always at least 1 link. But looking it carefully, at the hour 37, there is a local deviation to 0 links. This requires a more accurate analysis decreasing the time-step. For 30 seconds time-step:

\begin{figure}[H]
\begin{center}
\includegraphics[scale=0.30]{50_2,5_60_(30s)_lat.jpg}
\caption{Links vs time for latitudes from 50º to 60º with 30 seconds time-step}
\end{center}
\end{figure}

\paragraph{}
This analysis discard the latitudes 50º and 60º and reduces de range between 52.5º and 57.5º. 
\paragraph{}
In order to verify the equivalence between positive and negative latitudes -57º and 57º are compared with a more time-accurate analysis:

\begin{figure}[H]
\begin{center}
\includegraphics[scale=0.30]{-57_57_lat.jpg}
\caption{Links vs time for latitudes -57º and 57º}
\end{center}
\end{figure}

\paragraph{}
With the figure 1.7 it is verified that the coverage in a latitude is the same of the opposite one with a delay of 12 hours.
\paragraph{}
In conclusion, the optimum latitudes for the Ground Station are:
\begin{itemize}
\item Between -57.5º and -52.5º
\item Between 52.5º and 57º
\end{itemize}

\section{Longitude analysis}
It is intuitive to think that the effect of changing the longitude is delaying the evolution of the coverage. This efect is verified by the algorithm:

\begin{figure}[H]
\begin{center}
\includegraphics[scale=0.30]{0_90_180_270_long.jpg}
\caption{Links vs time for longitudes from 0º to 270º}
\end{center}
\end{figure}

\paragraph{}
As it is seen in the figure 1.8 the delay has a reason of 3 hours for every 45º of longitude. 
\paragraph{}
This effect can be used in order to optimize the performance of the Ground Stations. During the day every station will have a peak and a valley in the coverage. Placing the stations with a relative longitude of 120º one to the other would ensure that when one is at the valley other one is at the peak:

\begin{figure}[H]
\begin{center}
\includegraphics[scale=0.30]{0_120_240_long.jpg}
\caption{Links vs time for longitudes of 0º, 120º and 240º at a latitude of 57º}
\end{center}
\end{figure}

\paragraph{}
In conclusion, the Ground Station should be at a relative longitude of 120º one to the other. It has to be taken in account that this analysis is done for stations at the same latitude. A Ground Station in a given latitude has the same coverage behaviour as an other one at the opposite latitude and 180º of longitude away.
\paragraph{}
To exemplify, the following configurations are equivalent:

\begin{center}
\begin{tabular}{|c|c|c|c|c|c|c|}
\hline 
 & \multicolumn{2}{c|}{GS1} & \multicolumn{2}{c|}{GS2} & \multicolumn{2}{c|}{GS3} \\ 
\hline 
 & Latitude & Longitude & Latitude & Longitude & Latitude & Longitude \\ 
\hline 
Option 1 & 55 & 0 & 55 & 120 & 55 & 240 \\ 
\hline 
Option 2 & -55 & 180 & 55 & 120 & 55 & 240 \\ 
\hline 
Option 3 & 55 & 0 & -55 & 300 & 55 & 240 \\ 
\hline 
Option 4 & 55 & 0 & 55 & 120 & -55 & 60 \\ 
\hline 
Option 5 & -55 & 180 & -55 & 300 & 55 & 240 \\ 
\hline 
Option 6 & -55 & 180 & 55 & 120 & -55 & 60 \\ 
\hline 
Option 7 & 55 & 0 & -55 & 300 & -55 & 60 \\ 
\hline 
Option 8 & -55 & 180 & -55 & 300 & -55 & 60 \\ 
\hline 
\end{tabular}
\end{center}

\section{Conclusion}
\paragraph{}
Summarizing the results of the analysis, for an optimum performance of every Ground Station, they should be at latitudes between -57.5º and -52.5º or between +52.5º and +57.5º. For a better performance of the system every Station should be 120º of longitude away of the other stations if their are at the same latitude or 60º of longitude if they are at the opposite latitude.
\paragraph{}
Taking in account the topography of the Earth there are proposed the following options (every color represent the options for one Ground Station):

\begin{figure}[H]
\begin{center}
\includegraphics[scale=0.65]{Options.jpg}
\caption{Options for placing the 3 Ground Stations.}
\end{center}
\end{figure}

\paragraph{}
Given this possibilities it has to be done a study of the legislation of the involved countries legislation. The candidate countries are Canada, Argentina, Chile, Falkland Islands (Islas Malvinas), United Kingdom, Denmark, Norway, Sweden and Russia. 

\end{document}