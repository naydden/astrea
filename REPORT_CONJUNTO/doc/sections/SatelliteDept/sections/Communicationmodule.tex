\subsection{Communication module}


%http://gomspace.com/index.php?p=products-s100
%http://www.tethers.com/SWIFT.html
%http://www.tethers.com/SpecSheets/Brochure_SWIFT_UTX.pdf
%http://www.cubesatshop.com/product/isis-txs-s-band-transmitter/
%http://www.cubesatshop.com/product/highly-integrated-s-band-transmitter-for-pico-and-nano-satellite/
%http://www.cubesatshop.com/product/isis-vhf-downlink-uhf-uplink-full-duplex-transceiver/
% https://ocw.mit.edu/courses/aeronautics-and-astronautics/16-851-satellite-engineering-fall-2003/lecture-notes/l21satelitecomm2_done.pdf
% http://www.ae.utexas.edu/courses/ase463q/design_pages/spring03/cubesat/web/Paper%20Sections/4.0%20Communication%20Subsystem.pdf
%Convertir bad to mbps: % http://www.calculator.org/property.aspx?name=data+rate

\paragraph{} The telemetry subsystem analyses the information of the ground station and other sensors of the satellite in order to monitor the on-board conditions. With this system, the CubeSat is able to transmit the status of the on-board systems to the ground station.
\paragraph{}The command and control subsystem (TT\&C) allows the ground station to control the satellite.
\paragraph{} Every Astrea satellite (AstreaSAT) of the constellation, will need to report its operating status to the ground and receive commands from the ground. TT\&C operations will usually be performed when the satellite flights over the coverage of the constellation ground station, but since the satellites are interconnected, there is the possibility to perform this operations via data relay links between satellites. As a collaboration with the communications department, S band frequency is chosen for TT\&C operations, since there is no need for high data rates, the lower band will significantly reduce the power consumption.

\paragraph{}  Communication to the ground will be perform with a NanoCom TR-600 transceiver module attached to AntDevCo Patch antenna, both configured for S band frequency communication.


