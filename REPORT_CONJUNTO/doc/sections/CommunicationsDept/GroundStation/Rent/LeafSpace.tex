\section{Rent the GS}
There are a lot of ground stations spared all over the world. In the following pages a list of them can be seen in order to know who owns  them and their location. 
\subsection{ESA Ground Stations}
\begin{itemize}
\item \textbf{Kiruna Station}

\item \textbf{Kourou Station}

\item \textbf{Maspalomas Station}

\item \textbf{Redu Station}

\item \textbf{Santa Maria Station}

\item \textbf{Villafranca Station}

\end{itemize}
\subsection{KSAT Ground Stations}
\begin{itemize}
\item \textbf{Svabard Satellite Station}

\item \textbf{Tromsø Satellite Station}

\item \textbf{Troll Satellite Station}

\item \textbf{Grimstad}

\item \textbf{Hartebeesthoek}

\item \textbf{Dubai}

\item \textbf{Mauritius}

\item \textbf{Singapore}

\end{itemize}
\subsection{NASA Ground Stations}
\begin{itemize}
\item \textbf{Alaska Satellite Facility}

\item \textbf{McMurdo Ground Station}

\item \textbf{Wallops Ground Station}

\item \textbf{White Sands Ground Station}

\end{itemize}
\subsection{SSC Ground Stations}
\begin{itemize}
\item \textbf{Clewiston Satellite Station}

\item \textbf{Esrange Satellite Station}

\item \textbf{Inuvik Satellite Station}

\item \textbf{North Pole Satellite Station}

\item \textbf{Punta Arenas Satellite Station}

\item \textbf{Santiago Satellite Station}

\item \textbf{South Point Satellite Station}

\item \textbf{Dongara Satellite Station}

\item \textbf{Yatharagga Satellite Station}

\end{itemize}
\subsection{Other GS}
\begin{itemize}
\item \textbf{Goonhill Earth Station}

\end{itemize}
\subsection{Contact with GS companies}
Some companies that own a Ground Station have been contacted in order to get some information about costs and conditions of renting their stations. However, is important to notice that no answer is given for this type of project (students project). Moreover, information is not available on the Internet. If the project goes ahead, more information could be given to these companies and a cost can be obtained, so the option of renting one of the above cited GS is not discharged. Nevertheless, a cost is needed to know if is better to rent the GS or to build one. To do so, a company named LeafSpace will be used.
\subsection{LeafSpace}
LeafSpace is an italian company which provides a GS network, specifically designed to exchange data with micro and nanosatellites in a fast and simple way. Their global distribution ensures a high visibility time for a wide range of orbits, allowing their customers to download massive amounts of data. \\
This means that LeafSpace lets customers use their GS to download data, but does not permit to rent them in exclusive, which is the main idea of this project. Due to the small amount of information existing, LeafSpace will be considered in order to get a first approximation and to develop an OWA to decide. 

\subsubsection{Features}
\paragraph{Antenna}
LeafSpace allows to receive data from VHF (137-144 MHz), UHF (400-402 MHz), S-Band (2.2-2.4 GHz) and X-Band (8.025-8.5 GHz), but only can transmit UHF (401-403 MHz) and S-Band (2.025-2.11 GHz). The polarization is RHCP/LHCP (Right and Left Hand Circular Polarization, respectively). The moduation and the protocol are totally configurable. The datarates depend on the bandwidth: for UHF, up to 100 Kbps; for S-Band, up to 30Mbps; and for X-Band, up to 100Mbps. 

\paragraph{Pricing}
The prices, expressed in euros/Mbyte, depend on the bandwidth too: for receiving, VHF 5, UHF 5, S-Band 0.4 and X-Band 0.1, while for transmitting it is UHF 20 and S-Band 2 (recall that they can only transmit in those two bandwidths). 

Nevertheless, it is also stated that customized subscriptions are available for missions with large data transfers and constellations. Then, it is highly probable that a better pricing can be achieved. 

\paragraph{Boost Performance}
Within 2017, 20 Ground Stations are scheduled to be implemented all around the World, ensuring a telecommunication service with a considerable increase of visibility time, together with a drastic reduction of communication latency for a wide range of Low Earth Orbits. 

\paragraph{Way of use}
Data management is achieved with a user-friendly web-based interface, along with cloud storage granting direct access to download data at any time. 

Since this is all granted by LeafSpace, there would be no need to develop the Ground Segment discussed before. 

\paragraph{Services}
It is claimed to be 24/7 full availability of downloaded data, API access for constellations management, full redundant cloud storage for up to 10 days, advanced levels of data encrypting on demand, automatic scheduling, upling and downling, ranging and tracking, and 24/7 alert service. 

\paragraph{Map}
In the following image there is the planification of Ground Stations to be built in the following years by LeafSpace. 

\includegraphics{LeafSpace.jpg}

\paragraph{Operation}
No information relative to operation is given. It is certainly stated that its working way is automatic. Despite so, some maintenance is surely required, though its cost is probably low.

