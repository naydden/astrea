
The Ground Station is an indispensable part of almost any space mission. Such is its importance that it can even be seen as a subsystem of the mission.
\newline

This subsystem compose the Ground Segment of the mission and will be responsible of the extraplanetary communications with the spacecrafts. Furthemore, it will operate as a telecommunication port, which means that it will work as a hub, connecting the satellites to the Internet.
\newline

In order to establish communication in such high distances ($\approx$ 600km for LEO) high bands radio waves are going to be used. This is a requirement that is going to conditionate the overall Ground Station architecture.

\begin{itemize}
\item Since radio waves are going to be used, communication is established only when the Satellite has the Ground Staion (from now on GS)  in its line-of-sight. That will affect the location. Moreover, the orbits of the satellites will affect the GS location as well. The GS should be placed in a way that it gets maximum coverage time. This point will be further explained.

\item Depending on the target band to cover, which is the one used by the satellites for ground segment communication, the GS parts will vary in shape, size and prize significantly.
\end{itemize}

To use a GS there are two possibilites: build or rent one. In order to know which of the possibilities is the best, in the following lines they will be explained giving some numbers about the cost, and then a decision will be taken. 
