\section{Spare Strategy}
\subsection{Introduction}
When building a satellite constellation with the target to provide global coverage communication relay between LEO satellites and between LEO satellites and the ground, it is crucial to avoid any deterioration of the service. In order to ensure that any possible fail from the satellites would not spoil the constellation operation for more than 6 hours; a spare strategy has to be done. Nowadays, four different types of spare strategies are known:
\begin{itemize} 

\item {Spare satellites in constellation}
\item {In-orbit spare} 
\item {Spare satellites in parking orbits} 
\item {Spare satellites on the ground} 

\end{itemize}
Each existing spare strategy is valid. Despite, depending on the enterprise prioprities the most suitable has to be choosen. In addition, the decision taken is related to the constellation flexibility to degrade the service to a lower performance level during a certain period and to its cost. 

\subsection{Spare Strategy Alternatives}

\textbf{Spare satellites in constellation:}
\newline
This configuration consists on designing the constellation to be \textit{ "overpopulated"}. As it sounds, this means that the system is established with \textit{extra} operative satellites already orbiting within the constellation. For instance, only two overpopulating configurations had been pictured: ovepopulated by one satellite or overpopulated by two satellites per orbital plane. 

\begin{itemize}
\item[-] \textsc{One extra Satellite:}
\newline
By adding an extra satellite to the primary design of the orbital plane configuration, one satellite failure is covered with little time delay to recover the plan. In this way, the constellation continues to work at maximum capacity after a short interruption and at a suitable cost. 
\item[-] \textsc{Two extra Satellites:}
\newline
Usually, by adding two extra satellites per orbital plane the reliability of the service achieves values around the 99.99\%. This configuration increases considerably the cost of the project and it is mainly neccessary in cases where the availability of the satellite is essential for the proper operation of the constellation.
\end{itemize}
Therefore, when designing an overpopulated constallation, the first decition to be made is the number of extra satellite per orbital plane. To guarantee the most optimal configutation a feasibility study is needed.  
\newline
\newline
\textbf{In-orbit spare:}
\newline
The main diference between this strategy and the previous one is that in this case spare satellites are not operative. So the idea is to put some spare satellites in a orbit close to the principal one of the constellation in order to avoid possible collisions between operative satellites and spares. 
\newline
\newline
A few things have to be taken into account when using this method. Firstly,even though the spare satellites are not operative, by being in orbit they deteriorate and by the time they are needed their operative lifetime and performability will not be such as the ones of brand new satellites.Secondly, as their are non-controlled satellites their orbital decay has to be predicted to be aware of possible collitions and avoid them. Thirdly, once any spare satellites is needed, it has to be able to do a two Hohmann transfer to achieve the performance orbit; the first one to reach a phasing orbit and the second one to end in the operational altitude.
\newline
\newline
\textbf{Spare satellites in parking orbits:}
\newline
By mading this choice it has to be assumed that the spare satellites can be keeped in parking orbit until they are needed. Two different option are valid: keeping the rocket in a \textit{"parking"} orbit and then try to send it to the corresponding orbit; or keeping it in in-orbit satellites parkings such as the ISS. The main drawback is that the performance takes a long time until the constellation is recovered and depending on the orbit parameters and the launcher it is not possible to use this strategy.  
\newline
\newline
\textbf{Spare satellites in parking orbits:}
\newline
The simpliest and easiest one; the only thing that has to be done is to build extra satellites. The spares will remain on ground when the constellation is launched. Only in case the structure collapses due to a satellites failure, an emergency launch will put the spares in orbit. Moreover, this method is expensive because every extra launch has a high cost and it can take weeks to recover the constellation performance. 
\newline
\subsection{Spare Strategy Selection}
From all those alternatives, two of them  are quickly discarded: in-orbit spares and in parking orbit spares. The first one is having a non-working satellite in orbit because not only the satellite has to be purchased, but also it has to be launched to a different orbit than the principal one. That fact will increase the cost of the launch or even worst it could create the necessity of an extra launch. Although, the satellites needs to reach the operative orbit and it is known that cubesats propulsion is not really powerful. Furthermore, this satellites might never be needed. So it is highly probable this investment to be a waist of money and sources and this are the main reasons why it has benn discarded.
\newline
\newline
The second is not available in the \textit{Astrea Constellation} case. On the one hand, the main parking in orbit will be the ISS which is at an altitude of 400km above the earth and the constellation is situated at among 550km above the earth. Knowing that, this option is immediately discarded. On the other hand, the Electron the rocket that will accomplish the mission to put the satellites in orbit cannot stay in parking orbit before arriving to its final destination. Definitely, the service cannot rely on this option.
\newline
\newline
Two possible spare strategies remain: pare satellites in the constellation or on ground. In spite deciding if both ones are useful or only one of them is, a feasibility study is done. The objective is analise the diferent kind of failure that have to be covered and determine how the constellation will collapse. Only after that the most suitable strategy method can be designed having as reference the alternatives presented above. 
\subsubsection{Feasibility Studies}
The following studies are based on the probability of failure of the satellites and how the different combinations of failure could become a critical or not for the constellation operations. Furthermore, the main parameters needed during this studies are the number of satellites per plane, the probability of failure of a single satellite and the number of planes. 
\newline
\newline
Let's quantify the parameters presented above. The probability of failure of the satellite during the first five years is about \textbf{FALLABILITAT} and there are 21 satellites per orbital plane per 8 orbital planes. As a result, it means the whole constellations contains 168 satellites which have a 95\% of probability not to fail. By doing quick calculations, firstly it could be assumed that around 8 satellites would fail during the mission. However, this are simple calculations with low reliability.
\newline
\newline
As CubeSat communicate between other satellites and ground, two different cases of failure are observed. On the one hand, the constellation has been designed to offer global coverage of the Earth's surface, so once a satellites fail its footprint disappear and a small hole appears. On the other hand, a sophisticated network of communication between satellites has been created, so once a Cubesat fails it creates a hole in the communications network and entails the creation of new paths which avoid the non-operative satellite. Therefore, the collapse situation for each state has to be determined.
\newline
\begin{itemize}
\item \textsc{Communication with ground:}
The main problem with ground communications is the appearance of a big gap in the footprint the satellites leave. This gap is formed by the failure of several neighboring satellites and has to be big enough to cut all communication with the corresponding ground station to consider the failure as critical. Nevertheless, even if the communication with a single ground station was not possible for a short period of time the information could be transmitted to another ground station with the only drawback that the travel time of the data increases.   
\item \textsc{Communication between satellites:}
Once the communication of a satellite fails, it creates a gap in the communication network. Although any satellite outside the constellation (referring to the ones that use \textit{Astrea} services) tries to communicate, it will not have any problem to communicate with neighboring CubeSat to the one it failed if it is necessary. So the real problem could appear when two or more neighboring CubeSats fail. Additionally, the failure of a single satellite is not critical at all for the constellation because the data can still travel through other paths. 

\end{itemize}

 