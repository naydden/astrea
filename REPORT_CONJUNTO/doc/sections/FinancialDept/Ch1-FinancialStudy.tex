\chapter{Financial Study}

The different departments have estimated the main costs of the project. It is high time to start performing a deep analysis on the economical solvency of the project. 

Up to this point, it is important to determine how this product will be sold, so as to quantify the benefits of the project and be able to determine some figures as the Pay Back Time. 

\section{Selling the product: university clients}

Firstly, it has been thought that the service offered has great academic interests. In fact, any student could build a satellite with a certain payload, send it to space and then receive data from the satellite at any time thanks to Astrea constellation. However, students usually do have not enough money to afford that kind of project on their own. Consequently, for aerospace students from all over the world to get access to the service, it has been decided to offer a special fare to Universities.  Basically, for a reasonable price the universities will have access to our service through an annual subscription. In this way any student of the respective university will have access too.
\newline
\newline
In order to study the viability of doing a university fare, an estimation of the possible universities that would want to use the services had been done. Fortunately, the list of universities that offer studies in the aerospace field goes back a total of 400 schools approximately. Nevertheless, it is highly improbable that all those colleges become clients because not all universities have the same sources or interests. Therefore, the following list presents the number of existing colleges having an aerospace degree in each continent.
\newline
\newline	
	\begin{table}[!h]
	\begin{center}
	\begin{tabular}{|c|c|}
	\bf{Continent} & \bf{Number of Universities}\\
	\hline 
	Europe & 124\\
	\hline 
	Asia & 138\\
	\hline 
	North America &  97\\
	\hline
	 South America & 18\\
	\hline 
	Australia & 8\\
	\hline 
	Africa & 12\\
	\end{tabular}
	\end{center}
	\caption{Table. List of Universities with Aerospace Degrees}
	\end{table} 	
\newline
\newline
By analyzing this information, it can be determined that the continents with countries with higher PIB have more colleges interested in the space field. It is noticed that Asia is the continent with more colleges because, even if it is mostly poor, it is so big that it has rich countries such as Japan, Korea or China and the United Arab Emirates. Moreover, Europe and North America are not so extensive but have a higher aerospace culture and interest. 
\newline
\newline
On the basis the service is affordable for many prestigious colleges and it permits to provide their students with the chance to improve their knowledge by doing their own experiments, it has been estimated that about 150 universities will end up contracting our service in the next years. As a result, it is considered that it is suitable to offer an special fare for universities.


\section{Table with financial data}

In order to perform the analysis on the economical solvency of the project, following there is a table which contains the main costs of the project, as well as the numerical operations that allow to calculate some important financial parameters, such as the Net Present Value (NPV), the Internal Rate of Retorn (IRR), the Simple Pay Back Time (PBT), the Updated Pay Back Time (UPBT) and the Breakeven Point (BP). From this data, some conclusions will be drawn.

[TABLA UNA VEZ SEA DEFINITIVA]

\section{Conclusions of the financial study}
As a result of a few iterations of this table, changing some parameters, it has been found that:

[PBT, IRR, NPV, ...]








