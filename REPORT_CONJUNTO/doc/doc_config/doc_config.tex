%------------------------------------------------------------------------------------------------
%	DOCUMENT CONFIGURATIONS
%------------------------------------------------------------------------------------------------

% DOCUMENT
%   - Properties
\documentclass[a4paper,11pt]{report} % Sets a two-sided A4 sized paper.
                                         % (Two-sided for LaTeX to distinguish between odd/even
                                         % pages while using headers and footers.)
\linespread{1.15}                         % Default space between lines
                                    % Margin sizes defined in headers_and_footers file.
\usepackage[nottoc,numbib]{tocbibind}    % Makes References section appear in the Table of contents (nottoc). Conuts the References like section.
\usepackage{lastpage}                    % To get the total number of pages.

%   - Sectioning
\usepackage{titlesec}                   % Extra features for sectioning
    % The following lines create: \subsubsubsection{}
    
    
%%%%%%%%% From here: creation of \subsubsubsection{} %%%%%%%%%
    \titleclass{\subsubsubsection}{straight}[\subsection]

    \newcounter{subsubsubsection}[subsubsection]
    \renewcommand\thesubsubsubsection{\thesubsubsection.\arabic{subsubsubsection}}
    \renewcommand\theparagraph{\thesubsubsubsection.\arabic{paragraph}} % optional; useful if paragraphs are to be numbered

    \titleformat{\subsubsubsection}
      {\normalfont\normalsize\bfseries}{\thesubsubsubsection}{1em}{}
    \titlespacing*{\subsubsubsection}
    {0pt}{3.25ex plus 1ex minus .2ex}{1.5ex plus .2ex}

    \makeatletter
    \renewcommand\paragraph{\@startsection{paragraph}{5}{\z@}%
      {3.25ex \@plus1ex \@minus.2ex}%
      {-1em}%
      {\normalfont\normalsize\bfseries}}
    \renewcommand\subparagraph{\@startsection{subparagraph}{6}{\parindent}%
      {3.25ex \@plus1ex \@minus .2ex}%
      {-1em}%
      {\normalfont\normalsize\bfseries}}
    \def\toclevel@subsubsubsection{4}
    \def\toclevel@paragraph{5}
    \def\toclevel@paragraph{6}
    \def\l@subsubsubsection{\@dottedtocline{4}{7em}{4em}}
    \def\l@paragraph{\@dottedtocline{5}{10em}{5em}}
    \def\l@subparagraph{\@dottedtocline{6}{14em}{6em}}
    \makeatother

    \setcounter{secnumdepth}{4}
    \setcounter{tocdepth}{4}
%%%%%%%%% Until here: creation of \subsubsubsection{} %%%%%%%%%


\let\oldsection\section % This line and the following force sections to start on odd pages.
\def\section{\cleardoublepage\oldsection}

\newcommand*{\blankpage}{%
    \vspace*{\fill}
        \begin{center}
            (This page was intentionally left in blank.)
        \end{center}
    \vspace{\fill}
}
\makeatletter
\renewcommand*{\cleardoublepage}{\clearpage\if@twoside \ifodd\c@page\else
\blankpage
\thispagestyle{empty}
\newpage
\if@twocolumn\hbox{}\newpage\fi\fi\fi}
\makeatother

% WRITING
%   - Math/Chemistry/...
\usepackage{amsmath}                % Mathematical features.
    % Section numbering
    \numberwithin{equation}{section}    % Equation numbering by section
    \numberwithin{table}{section}       % Table numbering by section
    \numberwithin{figure}{section}      % Table numbering by section
\usepackage{amssymb}                % Symbols.
\usepackage[version=3]{mhchem}                 % Chemical equation typesetting. i.e.: \ce{CO2 + C <=> 2CO}
\usepackage{siunitx}                % Simplifies the usage of values with units.
                                    % i.e.: "\SI{5.4}{kg·m^{-1}·s^{-2}}"
                                    % instead of "5.4 kg·m$ ^{-1} $·s$ ^{-2} $"
\usepackage{eurosym}                % Use EURO symbol (\euro)

%%   - Language
\usepackage[utf8]{inputenc}         % Required for the usage of characters like 'ñ', 'ú', ...
%\renewcommand{\figurename}{Figura}  % In captions, print "Figura" (in Catalan) instead of
                                    % "Figure" (English default name).
%\renewcommand{\tablename}{Tabla}
%\renewcommand{\contentsname}{Índice}
%\renewcommand{\refname}{Bibliografía}

%   - Text
%\let\oldsection\section             % This line and the nextone force sections to start always in new odd pages.
%\def\section{\cleardoublepage\oldsection} %
\usepackage[none]{hyphenat}         % [none] Prevents any hyphenation throughout the document.
                                    % (hyphenation -> "separació per síl·labes")
\sloppy                             % Forces wrapping at word boundaries by relaxing the interword space constraints.
%\usepackage{indentfirst}            % Forces indentation from paragraphs after a section.
                                    % (see notes 1 and 2)
\setlength\parindent{1cm}           % Removes all indentation from paragraphs.
%\newenvironment{paragraphs}{\setlength\parindent{1cm}}{\setlength\parindent{0cm}}
                                    % "\begin{paragraphs}" will create an environment where
                                    % indentation from paragraphs is activated.
\usepackage[font=footnotesize]{caption} % Captions
\setlength{\abovecaptionskip}{0pt}      %
\setlength{\belowcaptionskip}{0pt}      %
\usepackage{lmodern}                % Allow the use of font size larger tha 25pt
\usepackage[T1]{fontenc}            %

% GRAPHICS, TABLES AND OTHERS
\usepackage{graphicx}               % Required for the inclusion of images.
\usepackage{float}                  % Required for some float properties.
\usepackage{pdfpages}               % Required for the inclusion of PDF files.
\usepackage{multirow,longtable,array,booktabs,tabularx} % More features for tables.
\usepackage{enumerate}              % Gives the enumerate environment an optional argument which
                                    % determines the style in which the counter is printed.
\usepackage{enumitem}               % Extra options for "enumerate" lists
\usepackage{color}                  % Use some color options.
\usepackage{colortbl}               % Use some other color options.
	\definecolor{UPC_blue}{RGB}{67,142,197}
	\definecolor{lightgrey}{RGB}{166,166,166}
\usepackage{hyperref}               % Provides LeTeX the hability to create hyperlinks.

\usepackage{inputenc}         % It makes possible to use roman numbers for the pages before the first chapter and arabic for the rest of the document

\usepackage{booktabs}
\usepackage{xcolor,colortbl}
\usepackage{pstricks}
\usepackage{blindtext}
\usepackage{transparent}
\usepackage{etoolbox}
\usepackage{lscape}

%EVA STUFF

\usepackage{amsmath}
\usepackage{amsfonts}
\usepackage{amssymb}
\usepackage{graphicx}
\usepackage{eurosym}
\usepackage{wrapfig}
\usepackage{mathdots}
\usepackage{caption}
\usepackage{cite}
\usepackage{mathrsfs}
\usepackage{float}

\usepackage{import}

%   - NOTES
% (1) LaTeX implements a style that doesn't indent the first paragraph after a section heading. There are coherent reasons for this. Some typography rules state that first indent should be suppressed only after a centered title and that all other paragraphs must be indented.
% (2) We have already removed all indentation from paragraphs with the command "\setlength\parindent{0cm}", but when using the environment we've created "escrit" we set indent to 1cm. Then, if this environment is used right after a section the first indent will be suppressed unless we use the package "indentfirst".