\section{Propulsion}

There is a big risk of a collision with space debris while a spacecraft is operating in Low Earth Orbits. The Inter-Agency Space Debris Coordination Committee recommended to the United Nations (section 5.3.2 'Objects Passing Through the LEO Region'): "Whenever possible space systems that are terminating their operational phases in orbits that pass through the LEO region, or have the potential to interfere with the LEO region, should be de-orbited (direct re-entry is preferred) or where appropriate manoeuvred into an orbit with a reduced lifetime. Retrieval is also a disposal option.” and “A space system should be left in an orbit in which, using an accepted nominal projection for solar activity, atmospheric drag will limit the orbital lifetime after completion of operations. A study on the effect of post- mission orbital lifetime limitation on collision rate and debris population growth has been performed by the IADC. This IADC and some other studies and a number of existing national guidelines have found 25 years to be a reasonable and appropriate lifetime limit.” \cite{collisionLEO}

Thus, a proper propulsion system is needed both for maintaining the satellite's orbit and for de-orbiting after the mission's lifetime.
 
The two most interesting options that were considered when the thruster had to be chosen are presented below.These two thruster are among the most used in the aerospace industry for small satellites. The main difference between both are the thrust and the specific impulse. On the one hand, the BIT-1 thruster provides a lower thrust but with a high specific impulse. On the other hand, BGT-X5 thruster provides a high thrust, around 0.5 N but with a lower specific impulse.

\begin{longtable}{| l | c | c | }
\hline
\rowcolor[gray]{0.80}	\textbf{Brand and model} &  \textbf{Features}     & \textbf{Total price (\euro)}   \\
\hline
\endfirsthead

\rowcolor[gray]{0.85} \textbf{Propulsion} &  &  \\
	   ~Busek ion thruster BIT-1 & \makecell{Volume 1/2 U \\ High Isp (2150 s) \\ Low thrust (100 uN)} & 58000 \\
	   \hline
	   ~Busek BGT-X5 & \makecell{Volume 1 U  \\ High thrust (0.5 N) \\ High delta V (146 m/s)} & 50000 \\
	   \hline

\caption{Options studied for the propulsion system}
\label{propulsionoptions}
\end{longtable}

The following table \ref{thrusterfinal} shows the main parameters of the thruster chosen.

\begin{longtable}{| l | r |}

\hline

\rowcolor[gray]{0.60} \multicolumn{2}{|c|}{\textbf{BGT-X5}} \\

\hline

\hline
\rowcolor[gray]{0.75}	\textbf{PARAMETERS} &  \textbf{VALUE}   \\
\hline

\cellcolor[gray]{0.85} \textbf{Total thruster power} & 20 W  \\
\cellcolor[gray]{0.85} \textbf{Thrust} & 0.5 N \\
\cellcolor[gray]{0.85} \textbf{Specific impulse} & 225 s \\
\cellcolor[gray]{0.85} \textbf{Thruster Mass} & 1500 g \\
\cellcolor[gray]{0.85} \textbf{Input voltage} & 12 V \\
\cellcolor[gray]{0.85} \textbf{Delta V} & 146 m/s \\
\hline
\caption{Main features of BGT-X5}
\label{thrusterfinal}
\end{longtable}