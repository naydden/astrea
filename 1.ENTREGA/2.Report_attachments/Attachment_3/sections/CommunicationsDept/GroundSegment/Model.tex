\paragraph{}
In order to find the protocols that will rule the ground communications of the network, it has been studied 4 protocol models and suits. It has to be found the advantages and disadvantages of each one and then, assess which suits better to the porpoise. The models and suits are:
\begin{itemize}
\item OSI
\item TCP/IP
\item NetBEUI
\item IPX/SPX
\end{itemize}
\paragraph{} 
OSI and TCP/IP have a complex structure which it is ideal for large networks. Although, this complexity makes the protocols inefficient in small networks, they are optimum for large ones. On the other hand NetBEUI and IPX/SPX protocols are simpler and optimum in simple communications. They cannot work in a large network unless adding extensions. 
\paragraph{}
In the beginning of the network working, the ground segment would not be so large as could be the space segment. Since there wont be lots of Ground Stations (3 at the beginning, but it could increase) the big part of nodes will be the clients. If it is want to be versatile and adapt to the demand, it has to be implemented protocols of the TCP/IP suit or based in the OSI model. Although these 2 will be more difficult to implement and configure, it will be the better option to ensure a good coverage for the demand and a friendly use to the costumers.
\paragraph{}
The main difference between OSI and TCP/IP is that the first is a model and the second is a suit of protocols. OSI is structured in 7 layers, and it describes how these layers should work. It is a theoretical guide for build a protocol but nobody had never implemented a complete OSI protocol. TCP/IP is structured in 4 layers and it is formed by a family of protocols which can be used in this layers. In the practice almost every network is ruled by TCP/IP protocols. 
\paragraph{}
The decision between OSI model and TCP/IP suit could be resumed as making a protocol or use existing ones. The work of the ground segment is not really different of many existing systems, so the better option is to use the existing TCP/IP protocols. It has to be found which fits better to the system in every layer and adapt it if is need.
\paragraph{}
More information about this models and suits can be found in \cite{OSI}, \cite{TCPIP}, \cite{NetBEUI} and \cite{IPX}
