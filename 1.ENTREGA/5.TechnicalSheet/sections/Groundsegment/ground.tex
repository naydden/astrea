\definecolor{blue}{rgb}{0.2,0.5,0.8}
\chapter{Ground segment}
The ground segment is composed by the Ground Stations (GS) and the Mission Control Center (MCC), that allow the receiving of the information from the constellation to the Earth.\\
The placement of the different nodes of the Ground Segment is shown in the following map, toguether with the amount of links that are available in a given instant (except for the MCC, that no links with satellites are established.  
\begin{table}[H]
\begin{center}
\begin{tabular}{|l|l|p{3.5cm}|p{3.5cm}|}
\hline
\rowcolor[gray]{0.80} \textbf{Node}&\textbf{Country}&\textbf{Minimum available number of links}&\textbf{Maximum available number of links}\\
\hline
GS1&Canada&2&12\\
\hline
GS2&Falkland Islands&2&12\\
\hline
GS3&United Kingdom&2&12\\
\hline
MCC&Spain&-&-\\
\hline
\end{tabular}
\caption{Countries of location and available links}
\end{center}
\end{table}
The MCC is composed by a set of offices with good connection to the GS. The systems that compose the GS are exposed in the following table. 
\begin{table}[H]
\begin{center}
\begin{tabular}{|p{1.5cm}|p{2cm}|p{3cm}|p{3cm}|p{3cm}|}
\hline
\rowcolor[gray]{0.80} \textbf{System}&\textbf{Frequency range}&\textbf{Features}&\textbf{Purpose}&\textbf{Elements included}\\
\hline
Two S-band systems&2-4GHz&Half-duplex system: downlink and uplink capability&Housekeepink data/TT\& C \newline
Client data upload&
Transciever 
\newline
LNA
\newline
HPA
\newline
RF Limiter
\newline
RF Swith
\newline
RF Fuse
\newline
Rotors\\
\hline
Two X-band systems&8-12GHz&X-band downlink capacity&Client data download&X-band receiver
\newline
LNA
\newline
RF Limiter
\newline
RF Fuse
\newline
Rotor\\
\hline
\end{tabular}
\caption{GS's systems}
\end{center}
\end{table}