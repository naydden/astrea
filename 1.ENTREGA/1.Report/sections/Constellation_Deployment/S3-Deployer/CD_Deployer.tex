\subsection{Deployer}
The objective of this section is to give a brief explanation of what is a deployer and how it works. 
As introduced above, there must be an adaptor between the rocket and the satellite in order to ensure subjection during the flight, efficient organization of the space in the fairing and a correct separation during the injection maneuver. This duty falls on the deployer. It consists on a prismatic structure prepared to carry the CubeSat inside. When the desired orbit is reached, the deployer uncovers one of its faces so as to let the satellite leave. There is a spring in the bottom that provides a little push to ensure that the CubeSat separates from the rocket.

There are many types of deployers, some of them are designed for an specific type of mission. As stated before, Electron is compatible with the standard CubeSat deployers, hence, only this type is considered. Similar to the case of the launcher selection, almost all the enterprises don't show enough information on the internet to reach a reliable conclusion, thus, some of them are contacted. Only two answers are obtained, one from ISIS (ISIPOD Deployer) and GAUSS (GPOD deployer). POD stands for Pico-satellite Orbital Deployer. 

They both present similar characteristics, however there are some differences. The main characteristics of the two deployers are listed in the  \cite[Chapter 1, Section 1]{annex2}, also two pictures of them are shown there.

In order to reach a reliable conclusion, two issues must be taken into consideration. First, the CubeSats of the Astrea Constellation are equipped with thrusters which increase the length of the satellite, thus, the deployer chosen cannot be fully closed. As seen in the \cite[Chapter 1, Section 1]{annex2}, GPOD has accessible panels whereas ISIPOD is fully closed, hence, ISIPOD is not suitable for the needs of the Astrea constellation. This condition automatically rejects the ISIPOD, nevertheless, there is a second reason for choosing the GPOD, the enterprise ISIS does not show the prices of their deployers even when a request is sent. Without this information it is decided that it cannot be taken into account. The price for the GPOD  deployer is 16000 US dollars per unit. 