\subsection{End-of-Life Strategy}
The main objective is to determine the best strategy to implement at the end of the operational lifetime of the satellites forming the constellation. In this way, it is possible to avoid an increase in space debris and in the collision risk between satellites positioned in the same altitude band or nearby, which is an existing problem explained in detail in \cite[Chapter 1, Section 2]{annex2}.

In order to make a decision, it has to be considered that the constellation is compounded of very small satellites (3U CubeSats). Those kinds of satellites cannot contain high thrust systems, consequently, the controlled de-orbit is out of its range as it has been checked in  \cite[Chapter 1, Section 4]{annex2}. Moreover, due to the fact the replacement strategy has been designed so as to avoid the need of a quick de-orbit, to adopt a controlled de-orbit decay it is not necessary and the uncontrolled one is still being adequate. Finally, the fact that the constellation is placed at LEOs makes easier the application of the uncontrolled de-orbit because the perturbations present in this altitudes increase the satellites decay rate of approach. As a result, given all the stated reasons, it is decided to use the uncontrolled de-orbit. 