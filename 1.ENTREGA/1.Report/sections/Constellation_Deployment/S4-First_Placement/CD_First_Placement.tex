\subsection{First Placement}
The aim of this part is to explain the first placement of the constellation. It is divided in two parts, the first one is intended to give a first approach to the logistics involved in the first placement. The second one is focused on the manoeuvre required so as to deploy the satellites into orbit. 
\subsubsubsection*{First Placement logistics}
The objective of this section is to give a general idea of the first placement logistics. Although some temporal data is provided, it is a qualitative explanation, only to clarify the order in which the different elements must be purchased, assembled, transported, etc. % Since this section does not attempt to provide neither exact data nor exact procedures
Rocket Lab provides two Gantt diagrams which are included at  [{REF TO ANNEX 2. Section 4 - First Placement}] on which their launching procedure is explained.
%
%\begin{figure}{}
%    \centering
%    \begin{subfigure}[!ht]{0.8\textwidth}
%        \includegraphics[width=\textwidth]{./sections/Constellation_Deployment/S4-First_Placement/Images_S4/Picture_1_S4.png}
%        \caption{Launch Range Operations Flow/Schedule}
%        \label{fig:gantt1}
%    \end{subfigure}
%    \begin{subfigure}[!ht]{0.8\textwidth}
%        \includegraphics[width=\textwidth]{./sections/Constellation_Deployment/S4-First_Placement/Images_S4/Picture_2_S4.png}
%        \caption{Countdown Operations Flow}
%        \label{fig:gantt2}
%    \end{subfigure}
%    \caption{Launching Logistics RL}\label{fig:animals}
%\end{figure}

The constellation has 189 3U CubeSats distributed in 9 orbital planes. One of the conclusions stated in the Launching System section  is that the quickest way to deploy the whole constellation is by carrying out one launch per orbital plane, consequently, the first placement consists on 9 launches and all the logistics around them. Rocketlab is capable of launching once per week, therefore, the first placement takes 9 weeks. Due to the magnitude of the mission, the whole rocket is filled with Astrea satellites, hence, there is no need to share it with other missions. Also, Rocket Lab offers an online booking procedure to reserve a date, however, The Payload User's Guide (provided by Rocket Lab) recommends contacting directly with them in case of filling several rockets with a mission instead of booking online.  

Since the schedule of Rocket Lab is fixed, the logistics needed in order to deliver the payload on time are going to be explained starting from the launching day, going back in time until the first movements in Terrassa, where the satellites are assembled. 

The launching day is designed L henceforth, and all the other ones are referred to this one (eg. L-30d means 30 days before launching). 

Rocket Lab needs 28 days to prepare the payload, place it into the rocket and prepare the rocket itself. Thus, the CubeSats have to arrive at the Rocket Lab launching facilities the L-28d. The satellites are assembled in Terrassa, hence, they have to be brought to New Zealand. Due to the large amount of CubeSats, the chosen transport is sea transportation. The estimated time from Terrassa to New Zealand is 30 days, so the CubeSats have to leave Terrassa the L-58d. At this point, there is two options. First, the 189 satellites can be divided in groups of 21 (number of satellites in an orbital planes) and sent separately to New Zealand so that every group arrives 28 days before its departure. The other option is to send all 189 CubeSats at the same time so that they arrive 28 days before the first launching. Each option has its pros and its drawbacks. Option one does not need to store the satellites in Rocket Lab facilites, conversely, the logistics of carrying each group of satellites separately is complicated. Option two allows to assemble all the satellites and send them in one ship, however, once they arrive to their destination, they have to be stored somewhere until their departure day arrives. Option two is selected because it is simpler and it is more likely to not cause delays delivering the payload to Rocket Lab, in addition, it is concluded that sending 9 ships with one week separation is not as efficient as sending a single one. 

The estimated time of assembling the satellites is twelve months, consequently, they have to be ordered the Launch minus 423 days. 

As clarified above, it is important to remember that the stated times are an approximation and the goal of this section is to give a first idea of the order of the different actions. 

\subsubsection{1st Placement Maneuver}
Once the Constellation is designed, it is essential to plan a proper procedure to put it in orbit. The Constellation is configured in several planes and satellites in each plane which work and communicate together in order to give signal coverage around the globe to finally accomplish their final purpose: intercommunicate other satellites form our customers.

One of the purposes of the project is to ensure the system is able to provide partial service right from the very beginning of its life, that is since the first orbital plane is put into orbit. Therefore, along with the maneuvers required to separate satellites in a certain orbital plane, the order in which the planes are put into orbit will also be assessed in this section. This particular section is crucial as it describes how the constellation is born.

\subsubsection{In-Orbit Injection}
It wouldn't be fair to start without mentioning the spaceship that will bring the whole system to life, and this is no more and no less than the Electron, from Rocketlab USA in New Zealand. The Electron is able to carry 24 3U CubeSats at once. Since 21 is the number of satellites needed in 1 orbital plane, it will be able to put one orbital plane into orbit in just one launch using the procedure described in the upcoming paragraphs.

Before starting any procedure description, it is important to set a start point. The first consideration is that there are still no Astrea satellites orbiting the earth. Therefore it is the first orbital plane that will be put into orbit. It is also considered that the rocket loaded with the 21 satellites has already accomplished all necessary maneuvers after lift-off and has just been able to arrive at the satellite's orbit, that is, proper altitude above Earth and proper tangential velocity. Of course at this point only the 2nd stage of the initial Electron rocket remains. Moreover, this stage is the one responsible of carrying the payload along with every single deployer. Once the start point is set, it is possible to thoroughly describe the procedure.

At the very described moment the first CubeSat is deployed into its final orbit around the Earth, which is a circular orbit at $542 \,km$ above Earth's surface. In order to deploy the second satellite at a given phase separation from the first one, the rocket must enter into an elliptical orbit with a slower period. Adopting this procedure will allow the needed phase separation between satellites given the fact that after one revolution of the rocket around the Earth, the first satellite will have gone through one revolution and a fraction more. In other words, at the very moment the rocket passes through the initial point which is tangential to the satellite's orbit, the first deployed satellite will be phase-wise ahead of the rocket. Obviously, the elliptical orbit mentioned must be accurately computed in terms of the increments in speed required to enter into it. A more schematic explanation can be found in  [{REF TO ANNEX 2. Section 4 - First Placement}]

Having pointed all of the above, it would make no sense to proceed without thoroughly going through the calculations of every single one of the required parameters to perform the manoeuvre. The first thing to take into account is the number of satellites for orbital plane. A number of 21 satellites per plane has been established, thus, a separation of $360^\circ/21 = 17.14^\circ$ between satellites will have to be accomplished. The velocity of the satellites and the period of their orbit can be computed. [{REF TO ANNEX 2. Section 4 - First Placement}]

Astrea's main purpose when it comes to 1st placement is to provide service as quickly as possible. This means that the time it takes to put a plane into orbit is crucial. This time will be determined by the period of the elliptical separation orbit that the rocket uses between deployments and of course by the number of satellites in each plane. Since 21 are the satellites that need to be put in orbit, 21 elliptical orbits will be needed. Therefore the time needed for one orbital plane is $3200\,s + 21*T_r = 129,191.6\,s$ which means 35.9 hours.

\subsubsubsection*{Plane Order}
Planes are going to be placed consecutively from the first one to the last one. This way allows a growing wider range of communication during the time the constellation is being set up. The whole discussion on which would be the best way to set up the orbital planes on the first placement can be found in  [{REF TO ANNEX 2. Section 4 - First Placement}].