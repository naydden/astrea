\chapter{Social and Security Considerations}
The potential of the CubeSats is very high and they might be the future of satellites. Their low cost and the easiness to construct them, compared to large satellites, make them accessible to countries with fewer resources, universities and people in general, making them able to explore the space and to pursue different missions.

This project is based on the design of a satellite constellation dedicated to communications relay between LEO satellites and between LEO satellites and the ground. This project is helping to develop the CubeSat industry and its use and it will demonstrate that these small satellites can carry out different missions that were previously done only by large satellites, as for example the communication.

Currently, the constellations of CubeSats dedicated to the communication are in development and this is why this project, and the global coverage that it provides, could have a privileged place in this industry. The main commitment that this project has with the customers is to ensure that they will be able to communicate with any part of the world without problems. Another important aspect to consider is that the constellation will provide total privacity  to the costumers, ensuring that they make a correct use of it and avoiding that third people interfering in the communication. 

In relation to security, it must ensured the proper functioning of the constellation. To do this, it must be considered different factors, where CubeSats could be in danger. The launch stage is one of the most important, because it is where the mission has more probability to fail. In the [{REF TO ANNEX: Anex V. Section 14}] it can be observed the succes rate of orbital launches in the last 57 years. In 2014, there were a total of 92 unmaned launches and only 4 of them were failed. This indicates that the fail rate is only a 4,34 \%, which is very low. 

Once the constellation is in orbit, CubeSats can find dangers how colliding with other satellites, which is not probably due to the distance between satellites is around hundred of miles, or with space debris, whose movement is unpredictable. In order to avoid this space debris, a CubeSat can perform a Debris Avoidance Manoeuvre (DAM). The responsible to control these fragmentation debris is \textit{The United States Space Surveillance Network}.

Finally, the ground stations will have its operator, to control the operation of the installation, and a security system, to avoid intrusions.