The Ground Segment is an indispensable part of almost any space mission. Given its importance it can even be seen as a subsystem of the mission.

This subsystem is composed by Ground Stations (GS) and the Mission Control Centre (MCC) and will be responsible of the extra-planetary communications with the satellites. Furthermore, it will operate as a telecommunication port, which means that it will work as a hub, connecting the satellites to the Internet.

In order to establish communication in such high distances ($\approx$ 600km for LEO) high band radio waves are going to be used. This is a requirement that is going to condition the overall Ground Station architecture.
\begin{itemize}
\item Since radio waves are going to be used, communication is established only when the Satellite has the Ground Station in its line-of-sight. That will affect the location. Moreover, the orbits of the satellites will affect the GS location as well. The GS should be placed in a way that it gets maximum coverage time; this point will be further explained.
\item Depending on the target band to cover, which is the one used by the satellites for ground segment communication, the GS parts will vary in shape, size and prize significantly.
\end{itemize}

To use a GS there are two possibilities: building or renting one. In order to know which of the possibilities is the best, in the following lines they will be explained giving some numbers about the cost. First of all, a study about building the Ground Segment will we done, analysing the location of the GS and the MCC, the legal aspect, the costs and maintenance, and the initial investment necessary to build them. After that, an analysis about renting GS will take place. Finally, a decision will be made.