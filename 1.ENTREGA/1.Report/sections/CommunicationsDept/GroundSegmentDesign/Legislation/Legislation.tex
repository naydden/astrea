The legislation will determine the location of the three GS between the locations pre-selected in the previous section. This is done because all the places pre-selected are more or less equivalent, and to choose between them governamental easy will be used. After doing a research on the legislation of all the places where the GS could be placed, only two countries have available legislation: Canada and United Kingdom. For this reason, the location for the 3 Ground Stations are United Kingdom, Falkland Islands and Canada. Falkland Islands are administered by United Kingdom, so the same license must be requested.

\subsection{United Kingdom Ground Station}
Non-Geostationary Earth Stations (Non-Geo). A Non-Geostationary Earth Station is a satellite earth station operating from a permanent, specified location for the purpose of providing wireless telegraphy links with one or more satellites in non-geostationary orbit. Therefore, this is the license required for United Kingdom and Malves Islands.

The form required to ask for the license can be found at \cite{UKForm}. The fees can be obtained from \cite{UKFees} and \cite{UKMHzFees}. The frequency allocation can be found in \cite{UKAllocation}.

\subsection{Canada Ground Station}
The Minister of Industry, through the Department of Industry Act, the Radiocommunication Act and the Radiocommunication Regulations, with due regard to the objectives of the Telecommunications Act, is responsible for spectrum management in Canada. As such, the Minister oversees the development of national policies and goals for spectrum resource use and ensures effective management of the radio frequency spectrum.

In Canada, the fees vary depending on the zone. There are three zones:
\begin{itemize}
\item High Congestion Zones: There are six metropolitan areas of Canada designated as zones of intense frequency use. They are in and/or around the following cities: Calgary, Edmonton, Montréal, Toronto, Vancouver and Victoria.
\item Medium Congestion Zones: There are 21 areas of Canada designated as zones of moderate frequency usage. These zones can be either stand-alone areas or areas that are adjacent to the six intense frequency use zones listed above. These moderate zones are as follows: Calgary, Chicoutimi, Chilliwack, Edmonton, Halifax, London, Montréal, Ottawa, the City of Québec, Regina, Saint John, Saskatoon, St. John's, Sudbury, Thunder Bay, Toronto, Trois-Rivières, Vancouver, Victoria, Windsor and Winnipeg.
\item Low Congestion Zones: These zones comprise all other areas of Canada.
\end{itemize}

It would be wise to choose a low congestion zone, which would have additionally less interferences. 

The process to fulfill can be found at \cite{CndForm}. The fees might be estimated using \cite{CndFees}.