\subsubsubsection*{Criteria for the orbital height of the satellites}
\begin{minipage}{0.45\textwidth}
If only geometric considerations were to be applied in the design of a satellite constellation, it is clear that the higher the orbit the broader is the footprint, leading to a smaller number of satellites. However, if the service of communications is to be offered, the satellites currently in orbit or in design phases need to be at higher orbit than the one of the constellation. The purpose of that requirement is to intersect the field of view of the satellites that nowadays point to Earth.
\end{minipage}
\begin{minipage}{0.5\textwidth}

\begin{figure}[H]
\centering
	\includegraphics[scale=0.6]{CurrentOrbitDistribution3}
	\caption[Distribution of the currently in orbit nanosatellites]{Distribution of the currently in orbit nanosatellites. Data on the 203 operative satellites from \cite{nanosats}}	
\end{figure}
\end{minipage}
\subsubsubsection*{The most interesting potential clients}
Lots of satellites are orbiting at heights lower than 500km, mainly because one of the most feasible way of launching a small satellite is from the International Space Station. However, this very low LEOs are related to very high speeds and specially to low lifetimes, since drag affects them in a more significant way. To the interest of the constellation, the satellites at higher altitudes are a better commercial target, since they are going to be in orbit for longer missions.

\textbf{New Space: Adapting to new society needs}
In the future, the possibility of using the Astrea constellation to contact Earth can reduce the requirements for the antennas and AOCSs to communicate with ground, leading to a new level of resources for the satellite payload. That is just a way in which Astrea is in the New Space Generation. The Generation that brings space closer to mankind.

\textbf{In conclusion}, in the decision process one of the statistics considered with certain weight will be the following: the ratio of satellites at which the constellation will be able to provide service considering that nowadays all of them point down to Earth. 