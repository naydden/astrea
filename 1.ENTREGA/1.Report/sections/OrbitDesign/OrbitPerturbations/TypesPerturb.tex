\subsection{Introduction to Orbit Perturbations\cite{Vallado2007}}\label{TypesPerturb}
%LET'S GO TEAM!!! ONE LESS SECTION
%We are unstoppppable --> DAB DAB DAB

In this chapter it is seen how the designed orbit configuration varies in time due to external perturbation sources. While some of them can be neglected, there are other of major importance to the future of the constellation. A first classification of perturbations depending on the time in which their effects are present is the following:

\begin{itemize}
\item Secular terms (Sec): They depend on the semimajor axis, the excentricity and the inclination.
\item Short Period terms (SP): They depend on the anomalies, this leads to a strong variation in each period.
\item Long Period terms (LP): They depend on the argument of the periapsis or the ascendent node.
\end{itemize}

Even though most of the outter space is vacuum, there ideal models need to consider some factors that escape the typical two body problem. To enumerate, here is a typical list of the main perturbation sources:

Sources of perturbation:
\begin{itemize}
\item \textbf{Gravity Field of the Central Body:} due to the Earth's aspherical shape as seen in [REF TO ANNEX I.Section 4.1.2]. This perturbation will not be considered because it does not affect Astrea's constellations as shown in [REF TO ANNEX I.Section 4.3.1]
\item \textbf{Atmospheric Drag:} It is the perturbation caused by the remaining atmosphere. The study of the satellites orbit decay can be found in [REF TO ANNEX I.Section 4.3.3]. Even so, this effect is not taken into account because the satellites are equipped with thrusters.
\item \textbf{Third Body perturbations:} Perturbation computed in [REF TO ANNEX I.Section 4.1.4]
\item \textbf{Solar-Radiation Pressure:} Explained in [REF TO ANNEX I.Section 4.3.2]
\item \textbf{Other Perturbations}
\end{itemize}
