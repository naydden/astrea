\chapter{Conclusions}

First and foremost, it is important to mention that the team has managed to accomplish the requirements of the client, which have been very demanding and challenging. All this would have not been possible without a robust team-work structure as well as a good management strategy.

Before production, more iterations of the design process are needed in order to obtain the exact parameters of the final service. For instance, if a better constellation configuration was found, it would necessary redefine the economic studies. This modification would be similar to the one that suffered the constellation Iridium \cite{Iridium}, when swapping from 77 to 66 satellites.

The future of the project is really exciting because of its definition and philosophy. Since every five years the constellation is going to be replaced by new satellites,i.e., new technology will be adopted much faster. Moreover, the growth of the number of nano-satellites is exponential. Hence, a communications constellation will be necessary in the following decades.

Within the NewSpace movement, Astrea aims to provide an FBC solution, i.e., faster, better and cheaper. This challenged the project to develop a solution in relatively short period of time.  

A competitive and realistic price has been set taking into account the offered service and the available and potential demand. The different economical indicators obtained by the result of the feasibility study have shown the attractiveness of the service.

\section{Recommendations}
The benefits of CubeSats Standards can be limiting in some aspects. A way in which we could improve the feasibility of the constellation is developing our own spacecraft subsystems instead of acquiring commercial options. This would improve the lifetime of the satellites and the replacements could be done with a lower frequency. The improvement would have in consideration the use of cutting edge technology philosophy.

Spacial Coverage to satellites is offered, providing the data-rate of \textbf{25 Mbit/s} per node. However, another great strategic advantage could be the development of a K-band system of communications for contacts between satellites, this means, communications in a frequency between $18-26.5 GHz$. This frequency band allows an order of magnitude of higher Data Rate. 
\section{Lesson Learnt}
{\footnotesize
\begin{itemize}
\item \textbf{Boyan Naydenov}: To coordinate such a big team has been a big challenge for me. Fortunately, its members have been amazingly hard-working so we were able to achieve great results together! Besides, I have seen the importance of coordination in a project of such calibre. A space engineering project!

 
\item \textbf{Josep Maria Serra}: Being part of this project has taught me the difficulty of agreeing on how to establish the format of the documents in a big group, and how things turn out smoothly when everyone knows their tasks.

 
\item \textbf{Oscar Fuentes}: This project has helped me obtain a deeper understanding in orbital mechanics and a valuable experience in teams management and, for the first time, coordination between teams.

 
\item \textbf{Josep Puig}: Astrea's planning and birth has helped me learn not only lots of technical space-related knowledge, but also to discover on how to properly perform team work, organize to elaborate such a big project and not dying during it, and also lots of details concerning project's feasibility study.

 
\item \textbf{Lluis Foreman}: Working with this team has had a direct impact on how I conceive team work and the importance of delegating tasks. This satellite based project has a lot of technical complexities which were well developed by each department, and between departments. I personally believe that this project is a success considering the difficulties that we encountered. 

 
\item \textbf{Victor Martinez}: By working on such a large team, I have realized how important it is to keep good communication and coordination in order to succeed. In addition, being part of this project, has given me the opportunity to learn and understand many concepts related to orbital mechanics.

 
\item \textbf{Joan Cebrian}: By working on this project I have learnt some important lessons such as team organization, decision taking and a lot more. I have been able to see better ways to coordinate a team in order to divide tasks. I believe all the stuff learnt here will be very useful in future projects. Finally, I find the feasibility studies very valuable given that it's a very important issue in engineering. 

 
\item \textbf{Roger Fraixedas}: It's been the first time for me to take part on such an extraordinary project. This experience has let me realise many important things for my future as an engineer. Among those, the importance of dividing one complex problem into as many simpler problems as possible. Without doing so it wouldn't be possible to come up with a sensible solution. In addition to that, the importance of planning ahead the scheduling in which different tasks must be completed. Overall I believe that the whole thing has resulted in such a success taking into account all the difficulties we've encountered.

 
\item \textbf{Marina Pons}: It has been the first time I have had the opportunity to work on such a huge project. As many things, I learnt working with 17 people has its benefits and drawbacks. Firstly, it was difficult to take decisions due to the lack of organization. However, by establishing a hierarchy and with the methods learnt to divide and schedule tasks, it has been possible to profit at maximum all the team resources. Finally, I concluded that without good coordination is very difficult to make big things happen.

 
\item \textbf{David Morata}: I had never worked on such a huge project until this last semester. Being involved in the design and the study of the feasibility of a project like \textit{Astrea} has been a really enriching experience to me. I have realized that working with so many people is a difficult task and some coordination is mandatory to deliver everybody's tasks on time. Now I can imagine how difficult can it be to get things done for agencies such as NASA or ESA. Overall, it has been a very positive experience. 

 
\item \textbf{Pol Fontanes}: Astrea constellation  its been my first approach to a real space engineering project. Space systems are really multidisciplinary, working on the satellite design, I have learned how to dimension and integrate all the subsystems for a spacecraft. Because of this degree of complexity, I gained a new point of view about managing tasks, delegating work is essential for meeting deadlines.

 
\item \textbf{Eva Maria Urbano}: Work in a group of 17 people may seem difficult. Well, it's difficult. However, with a proper coordination things got easier than I thought. I also thought about what I had to do: study and select protocols (among other tasks). It sounded a tedious task to do but when you start to understand what protocols mean and how they achieve things, they are quite funny. All in all, I am proud of the whole group and the project done.

 
\item \textbf{Fernando Herran}: With this project I have learned that to work in such a big team, it is necessary to have a good organization and communication in order to advance together in the same direction and  achieve the goals. In addition, this project has allowed me to understand better the operation of a satellite and the different elements that compose it.

\item \textbf{Silvia Gonzalez}: This project has been a great opportunity to learn about space systems and orbit mechanics. Also it has presented a huge challenge in terms of team organization and communication. Despite this difficulties, I am proud of how this hard-working team has manage to obtain amazing results.

\item \textbf{Laura Pla}: This project made me realize how important is the communication and organization in a big team.  Coordination between the different departments has been essential to divide the tasks and achieve all the objectives of the team. It has also helped me to learn more about constellations and orbital mechanics.

\item \textbf{Xavier Tió}: I've always heard that a great amount of the time and resources invested in a big project go to management and organisation. However, I didn't have the chance to prove it until this project. Working with a lot of collegues is always challenging and this project made us improve  in this aspect. Besides, we learnt how to deal with deadlines and with decision taking discussions. 

\item \textbf{Sergi Tarroc}: The main thing that I have learned in this project is how important is the organisation and the well communication in the team. We have invested lot of time in this issue, and in the moment could seem slow and inefficient, but in long term is really important. We've made a lot of work, but when the organisation have failed, the individual efforts have been useless. 

\end{itemize}
}