\chapter{State of the art}
The objective of this chapter is to make an overview of the current state of the technology covered by this project. 

Constellations of satellites are not a new concept. One of the most known examples is Iridium \cite{Iridium}, a global constellation of 66 low Earth orbit satellites that provides voice and data connections on the entire surface of the planet. The problem that the company had is that when the constellation was in orbit, the cellular technology on Earth was better and much cheaper than the service the constellation was offering. However, it is still operative nowadays, because there are still regions of the Earth that are not covered by any other service.

There are other services similar to the one that Astrea wants to offer but not based on constellations. As discussed previously, one of the problems of satellites is that their connection is limited to the amount of time that they are above a ground station. Leaf Space is an Italian startup company that plans to build a network of ground stations to provide service to nano, micro and small satellites in order to increase their visibility time. However, there are other companies that are currently offering this service to small satellites, such as Spaceflight Networks or Kongsberg Satellite Services (KSAT), but they are more focused on multiple satellites missions or constellations. The problem that ground station networks have is that they will never be able to provide full visibility time of the satellites because there are places in which a ground station can not be place, just as the sea.

In the recent years, the raise of CubeSats has let to the research of new possibilities for these small satellites, such as constellations. One of the examples is the QB50 Project, a mission held by the Von Karman Institute for Fluid Dynamics, that wants to facilitate the access to space through a constellation of CubeSats \cite{qb50}. There are also planned missions, like the BIRDS constellation from the Kyushu Institute of Technology. It consists of five 1U CubeSats, that are going to do experiments on radio communications \cite{KyushuInstituteofTechnology2017}. Another project is the one proposed by Planet Labs, a startup of former NASA employees. They are planning to launch a constellation of 3U CubeSats designed by them called Doves. These satellites will continuously scan the Earth, providing updated images of the planet.

Another most advanced project is Spire, a constellation of CubeSats that collects data of the Earth through a network of small satellites \cite{spire}. It has succesfully launched twelve low Earth orbit satellites.

However, the most similar project to Astrea is the communications constellation proposed by Kepler Communications \cite{keppler}. This company founded by graduate students from the University of Toronto, plans to provide Internet of Things for devices on the Earth and in orbit.

Despite all the constellations, there are no precedents in a constellation that communicates with satellites other than the ones of its own network.