\chapter{State of the art}
The objective of this chapter is to make a overview of the current state of the technology covered by this project. 

\section*{CubeSats}
The main reference of this information is \cite{SOTA}, a complete analysis done by NASA which describes the state of the art of the small spacecrafts technology and how it is expected to develop.
\subsection*{Power}
Driven by weight and size limitations, small spacecraft are using advanced power generation and storage technology such as $>29\% $ efficient solar cells and lithium-ion batteries. The higher risk tolerance of the small spacecraft community has allowed both the early adoption of technologies like flat lithium-polymer cells as well as commercial-off-the-shelf products not specifically designed for spaceflight. This dramatically reduces cost and increases flexibility of mission design. However, despite these developments, the small spacecraft community has been unable to use other, more complex technologies. This is largely because the small spacecraft market is not large enough to encourage the research and development of technologies like miniaturized nuclear energy sources.
\subsection*{Propulsion}
A significant variety of propulsion technologies are currently available for small spacecrafts. While cold gas and pulsed plasma thrusters present an ideal option for attitude control applications, they have limitations for more ambitious maneuvers such as large orbital transfers. Other alternatives such as hydrazine, non-toxic propellants and solid motors provide a high capability and are suitable for medium size buses and missions that require higher $\Delta$V budgets. Some spacecrafts have already flown with these systems or are being scheduled to fly in the next year. For the near future, the focus is placed on non-toxic propellants that avoid safety and operational complications and provide sufficient density and specific impulse. The application of this technology in cubesats is still in development as some of the components need to be scaled down to comply with volume, power and mass constraints.
Electrosprays, Hall Effect thrusters and ion engines are in an active phase of development and active testing and technology demonstrations are expected for different bus sizes. These propulsion technologies will allow spacecrafts to achieve very high $\Delta$V and, therefore, to perform interplanetary transfers with low thrust.
\subsection*{Structures, Materials and Mechanisms}
The landscape for small spacecraft structural design is expanding and the firms developing and offering solutions for spacecraft designers are expanding as well. Most of the developments have been in the 3U CubeSats and there are now some examples of designs for 6U and 12U CubeSats.
\subsection*{Thermal Control System}
As thermal management on small spacecrafts is limited by mass, volume and power constraints, traditional passive technologies, such as MLI, paints, coatings and metallic thermal straps, still dominate thermal design. Active technologies, such as thin flexible resistance heaters have also seen significant use in small spacecraft, including some with advanced closed-loop control. Passive louvers and sun shields have been proposed and developed for small spacecraft and will tentatively fly in 2016 (Dellingr and CryoCube-1). Deployable radiators and various types of composite thermal straps have also been fabricated and tested for small spacecrafts. Thermal storage units are being developed in order to have better control of the amount of heat dissipation.

\section*{Communications}
There is already a strong flight heritage for many UHF/VHF and S-band communication systems for CubeSats. X-band systems are less common but its use is growing in the last years. The use of even higher RF frequencies and laser communications already has been implemented on CubeSats, but with limited performance. Ka-band systems are currently in development, but are still low matured. On the other hand, laser communication is technology that will most likely see increased performance in the near future for onboard laser systems.

\section*{Integration, Launch and Deployment}
There is a wide variety of integration and deployment systems. While leveraging excess payload space will continue to be profitable into the future, dedicated launch vehicles and new integration systems are becoming popular to fully use the advantages provided by small spacecrafts. Dedicated launch vehicles may be used to take advantage of rapid iteration and mission design flexibility, enabling small spacecrafts to decide the mission parameters. Advanced systems may be used to host secondary payloads on orbit to increase mission lifetime, expand mission capabilities, and enable orbit maneuvering.

\section*{Ground Data Systems and Mission Operations}
Depending on the requirements and priorities of the user, different types of solutions to build and assemble a ground station are available in the market. If the user wants to focus more on the payload and the system engineering of the spacecraft, some companies have pre-defined solutions, which provide full capability and support for the spacecraft-ground communications. Other possible solutions are customizing the ground station with specific components (such as antennas, transceivers, modems and software) provided by different manufacturers. Finally, another valuable solution for small spacecrafts to communicate with Earth is using inter-satellite communications relays. Some CubeSat missions have already demonstrated these capabilities. 

\section*{Constellation of CubeSats}
The raise of these small satellites has let to the research of new possibilities for CubeSats, such as satellite constellations. This concept has been discused for the last years. One of the examples is the QB50 Project, a mission that wants to facilitate the access to space through a constellation of CubeSats \cite{qb50}. There are also planned missions, like the BIRDS constellation from the Kyushu Institute of Technology. It consists of five 1U CubeSats, that are going to do experiments on radio communications \cite{KyushuInstituteofTechnology2017}. However, the most interesting case is the already operative Spire. It is a constellation of CubeSats that collects data of the Earth through a network of small satellites \cite{spire}.

Despite all the constellations, there are no precedents in a constellation that communicates with satellites other than the ones of its own network.