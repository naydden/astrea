\chapter{State of the art}
The objective of this chapter is to make a overview of the current state of the technology covered by this project. The main reference of this information is \cite{SOTA}. It is a complete analysis done by NASA which describes the state of the art of the small spacecraft technology and how this technology is expected to develop. 

\section{Integrated Spacecraft Platforms}
On one hand a some vendors have pre-designed fully integrated cubesat, available for purchase. Due to the small market they have to cooperate with customers to customize the platform. On he other,  due to the cubesat standard interfaces, many standardized components are available, leveraging consumer electronics standards to approach the plug and play philosophy available for terrestrial PCs and computer servers. At present software is lagging behind hardware in modularity and reusability, and represents the largest hurdle to delivering cubesat missions. 

\section{Power}
Driven largely by weight and size limitations, small spacecraft are using advanced power generation and storage technology such as $>29\% $ efficient solar cells and lithium-ion batteries. The higher risk tolerance of the small spacecraft community has allowed both the early adoption of technologies like flat lithium-polymer cells as well as commercial-off-the-shelf products not specifically designed for spaceflight. This dramatically reduces cost and increases flexibility of mission design. 

Despite these developments, the small spacecraft community has been unable to utilize other, more complex technologies. This is largely because the small spacecraft market is not yet large enough to encourage the research and development of technologies like miniaturized nuclear energy sources. Small spacecraft power subsystems would also benefit from greater availability of flexible, standardized power management and distribution systems so that every mission need not be designed from scratch. In short, today’s power systems engineers are eagerly adopting certain innovative Earth-based technology – like lithium polymer batteries.

\section{Propulsion}
A significant variety of propulsion technologies are currently available for small spacecraft. While cold gas and pulsed plasma thrusters present an ideal option for attitude control applications, they have limitations for more ambitious maneuvers such as large orbital transfers. Other alternatives such as hydrazine, non-toxic propellants and solid motors provide a high capability and are suitable for medium size buses and missions that require higher $ \bigtriangleup v $ budgets. Some spacecraft have already flown with these systems or are being scheduled to fly in the next year. For the near future, the focus is placed on non-toxic propellants that avoid safety and operational complications and provide sufficient density and specific impulse. The application of this technology in cubesats is still in development as some of the components need to be scaled down to comply with volume, power and mass constraints.

Electrosprays, Hall Effect thrusters and ion engines are in an active phase of development and active testing and technology demonstrations are expected for different bus sizes. These propulsion technologies will allow spacecraft to achieve very high $ \bigtriangleup v $ and,therefore, to perform interplanetary transfers with low thrust.


\section{Guidance, Navigation and Control}
Small spacecraft Guidance, Navigation and Control (GNC) is a mature area, with many previously flown components, offered by several different vendors. Soon-to-be-flown integrated units will offer a simple, single vendor single component solution for Altitude Determination and Control Subsystem (ADCS) which will simplify GNC subsystem design. 

\section{Structures, Materials and Mechanisms}
The landscape for small spacecraft structural design is expanding and the firms developing and offering solutions for spacecraft designers is expanding as well. Most of the developments have been in the 3U cubesat class and there are now at least a few examples of mature structural designs for 6U class cubesats, with 12U designs being presented for future standardization. 

There have already been some very interesting uses of 3D printed materials, and it appears that the application of these materials for space flight missions is on the very near horizon, including exploiting for purpose-built radiation shielding. Whether or not the promised benefits of these materials outweigh those of more conventional materials in the near future remains to be seen. 

\section{Thermal Control System}
As thermal management on small spacecraft is limited by mass, volume and power constraints, traditional passive technologies, such as MLI, paints, coatings and metallic thermal straps, still dominate thermal design. Active technologies, such as thin flexible resistance heaters have also seen significant use in small spacecraft, including some with advanced closed-loop control. Passive louvers and sun shields have been proposed and developed for small spacecraft and will tentatively fly in 2016 (Dellingr and CryoCube-1).Deployable radiators and various types of composite thermal straps have also been fabricated and tested for small spacecraft utilization in the past few years and are offered from numerous vendors. Cryocoolers are not enough developed to integrated them in cubesats. Thermal storage units are being developed that will better control amount of heat dissipation as well as storing energy for future use. 

\section{Command and Data Handling}
As cubesat development and application continues to evolve, there are a wide range of avionics systems and components available to address the needs of the wide range of small spacecraft developers, professional and amateur. Complete avionics packages are available to those who seek an integrated solution. At the other extreme, open source DIY kits are available to those who seek a low cost way to explore developing their own Command and Data Handling system and spacecraft. 

\section{Communications}
There is already strong flight heritage for many UHF/VHF and S-band communication systems for cubesats. Less common but with growing flight heritage are X-band systems. The use of even higher RF frequencies and laser communication already has some flight heritage on cubesats, but with limited (or yet to be demonstrated) performance. Ka-band systems for cubesats are currently in development, but it is still low matured. On the other hand, laser communication is a spaceflight ready technology that will most likely see increased performance in the near future for onboard laser systems. Alternatively, a few groups are working on asymmetric laser communication, but it is still a relatively low mature technology.

\section{Integration, Launch and Deployment}
A wide variety of integration and deployment systems exist to provide rideshare opportunities for small spacecraft on existing launch vehicles. While leveraging excess payload space will continue to be profitable into the future, dedicated launch vehicles and new integration systems are becoming popular to fully utilize the advantages provided by small spacecraft. Dedicated launch vehicles may be used to take advantage of rapid iteration and mission design flexibility, enabling small spacecraft to dictate mission parameters. New integration systems will greatly increase the mission envelope of small spacecraft riding as secondary payloads. Advanced systems may be used to host secondary payloads on orbit to increase mission lifetime, expand mission capabilities, and enable orbit maneuvering. In the future these technologies may yield exciting advances in space capabilities.

\section{Ground Data Systems and Mission Operations}
Depending on the requirements and priorities of the user, different types of solutions to build and assemble a ground station are available in the market. If the user wants to focus more on the payload and the system engineering of the spacecraft, some companies have pre-defined turnkey solutions, which provide full capability and support for the spacecraft ground communications. Other possible solutions are customizing the ground station with specific components (such as antennas, transceivers, modems and software) that can be provided by different manufacturers.  Finally, another valuable solution for small spacecraft to communicate with Earth is using inter-satellite communications relays. Some cubesat missions have already demonstrated these capabilities. 

\section{Passive Deorbit Systems}
Small spacecraft deorbit systems are relatively immature but are necessary to meet space debris mitigation requirements. As most small spacecraft are unable to relocate to a graveyard orbit due to propulsion limitations, deorbit system development has focused on passive devices. NanoSail-D2,DeorbitSail and CanX-7 are all cubesat platforms that have successfully demonstrated the utilization of drag sails for deorbiting in Low Earth Orbit within the 25 year post mission requirement. Terminator Tape is another deorbit option that uses electromagnetic tethers that is currently being flown on Aerocube-V cubesat. 