\section{Propulsion Systems}
The propulsion system is an important part of the satellite given the needs of the satellite to perform different maneuvers, such as reach the desired orbit after it has been deployed from the rocket, and to maintain the orbit and avoid falling to the Earth. The main parameters that have to be consider are thrust, total specific impulse, power required, weight of the propulsion system and its volume, since the size and weight of the CubeSat are very restrictive. [{REF TO ANNEX IV. Section 1.3}]

At the moment, the most used and modern thrusters for small satellites are: ionic, pulsed plasma, electrothermal and green monopropellant thrusters. An important aspect to consider is that the goal is to reduce the mass required although this will imply smaller accelerations than conventional propulsion systems.

The \textbf{BGT-X5}, a green monopropellant propulsion system, has been chosen as the thruster for the CubeSat. The high thrust and delta V that BGT-X5 provides (146 m/s) complies with the requirements explained in [{REF TO ANNEX Const. Deployment}] and they have predominated at expenses of other variables as weight or specific impulse where others options were better. With this thruster the CubeSat will be able to carry out the necessary actions to keep the satellite in orbit, to relocate the satellite or to change its orbit. [{REF TO ANNEX IV. Section 1.3}]

The option chosen is presented in the table below \ref{propulsionfinal}.

\begin{longtable}{| l | r | r | }
\hline
\rowcolor[gray]{0.80}	\textbf{System} &  \textbf{Brand and model}     & \textbf{Price per unit (\euro)}   \\
\hline
\endfirsthead

	   ~Propulsion & Busek BGT-X5 & 50000 \\
	\hline

\caption{Option chosen for the propulsion system}
\label{propulsionfinal}
\end{longtable}