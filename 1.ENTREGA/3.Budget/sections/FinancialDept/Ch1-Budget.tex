\chapter{Introduction}
The aim of a budget is to provide reliable information about the estimation of the expenses required in order to start and carry out Astrea, as a new company.

Astrea's situation is kind of unusual, due to the fact that the company is not selling a defined physical product like a plane, but a service. To do so, a widespread constellation of cubesats is deployed around the whole Earth. This situation obviously requires of a certain maintenance and periodically a complete renewal. In other words, over the years, some other investments must be carried on. 

In particular, in the year 0 of Astrea's life (this is, the planning and beginning) there are many different costs such as the engineering hours, the communications' first investment, the satellites building and launching, amongst many others. Nevertheless, 5 years from that point (this is, year 4) the life of the satellites is expected to be getting over, and in order to prevent fails on the service, a replacement strategy is done. In year 4 new satellites will be built and in year 5 they will be re-launched, so as to fill the constellation again. This will mean that a second budget is done for the year 4, including both building and launching. Again, in year 9, the same phenomenom will occur, and so on. 

Therefore, two different budgets are going to be estimated. The first one is the initial budget, and the second one is the periodic budget, which must be invested every 5 years from the beginning of Astrea's life.  

In order to be conservative, all the figures in this document will be approximated to the high closest thousand value. 

All the figures in this document are \textbf{in \euro}, unless stated the opposite, apart from the final budgets, which are \textbf{in M\euro}. 

Without further ado, let's state the different costs of the project.


\chapter{Costs}
\section{Manpower}
The engineering hours were stated in the Project Charter together with the Gantt Chart. They are again synthesized in the following table. It must be taken into account that the salary of an Astrea's engineer is of \euro  per hour. 

\begin{longtable}{ccc}
\toprule
\rowcolor[gray]{0.75}
    \textbf{Engineering hours budget} & \textbf{Hours} & \textbf{Labor cost (\euro)} \\
    \midrule
    \endhead
\hline
\rowcolor[gray]{0.85}
	MANAGEMENT &  &  \\ \hline
	Meetings documentation &  &  \\ \hline
	Meetings & 340 & 6800 \\ \hline
	Meetings preparation &  &  \\ \hline
	Agendas & 10 & 200 \\ \hline
	Minutes & 10 & 200 \\ \hline
	Task Tracking and scheduling &  &  \\ \hline
	Project Charter & 170 & 3400 \\ \hline
	Team tasks monitoring & 20 & 400 \\ \hline
	WBS and Gantt update & 10 & 200 \\ \hline
	\rowcolor[gray]{0.85}
	SATELLITE DEVELOPMENT &  &  \\ \hline
	Spacecraft subsystems & 180 & 3600 \\ \hline
	Payload &  &  \\ \hline
	Antenna & 40 & 800 \\ \hline
	PHDS & 50 & 1000 \\ \hline
	\rowcolor[gray]{0.85}
	ORBITAL DESIGN &  &  \\ \hline
	Constellation geometry & 220 & 4400 \\ \hline
	Orbit parameters &  &  \\ \hline
	General parameters & 120 & 2400 \\ \hline
	Drift & 100 & 2000 \\ \hline
	Legislation & 50 & 1000 \\ \hline
	\rowcolor[gray]{0.85}
	LAUNCH SYSTEMS &  &  \\ \hline
	Vehicle & 60 & 1200 \\ \hline
	Satellite deployer & 10 & 200 \\ \hline
	Replacement strategy & 100 & 2000 \\ \hline
	\rowcolor[gray]{0.85}
	OPERATION &  &  \\ \hline
	Communication protocol & 100 & 2000 \\ \hline
	Ground station & 80 & 1600 \\ \hline
	End of life strategy & 80 & 1600 \\ \hline
	\rowcolor[gray]{0.85}
	FINANCIAL PLAN &  &  \\ \hline
	Costs &  &  \\ \hline
	Fix &  &  \\ \hline
	Maintenance and cost analysis & 10 & 200 \\ \hline
	Insurance cost analysis & 15 & 300 \\ \hline
	Administration cost analysis & 15 & 300 \\ \hline
	Taxes cost analysis & 25 & 500 \\ \hline
	Variable &  &  \\ \hline
	Manufacturing cost report & 10 & 200 \\ \hline
	Launching cost report & 10 & 200 \\ \hline
	Income &  &  \\ \hline
	Price analysis & 25 & 500 \\ \hline
	Revenue forecast & 25 & 500 \\ \hline
	Economic feasibility report & 40 & 800 \\ \hline
	Marketing Plan & 20 & 400 \\ \hline
	\rowcolor[gray]{0.85}
	PROJECT EXHIBITION &  &  \\ \hline
	Constellation simulation & 30 & 600 \\ \hline
	\rowcolor[gray]{0.65}
	TOTAL & 1975 & 395000 \\
	\bottomrule\\
	\caption{Engineering hours cost (manpower)}
\end{longtable}



\section{Communication Costs}
The communications costs include the costs of building the Ground Stations and the Mission Control Center, and also the costs of maintenance and operation of them.
\subsection{Initial Investment}
The investment required for building each Ground Station is of 356000 \euro and the Mission Control Center of 3000000 \euro. 


\begin{table}[h]
\begin{center}
\begin{tabular}{ | l | l | }
\toprule
\hline
\rowcolor[gray]{0.75}
	 & Cost \\ \hline
	GS Canada & 356000 \\ \hline
	GS Malvines & 356000 \\ \hline
	GS Scotland & 356000 \\ \hline
	MCC Spain & 3000000 \\ \hline
	\rowcolor[gray]{0.65}
	Total & 4068000 \\ \hline
	\bottomrule

\end{tabular}
\caption{Ground Stations and Main Control Center initial investment}
\end{center}
\end{table}




\subsection{Maintenance and Operation Costs}
The following costs are associated to operation resources and general maintenance:

\begin{table}[h]
\begin{center}
\begin{tabular}{ | l | l | l | l | l | }
\toprule
\hline
\rowcolor[gray]{0.75}
	Concept & MCC & GS Canada & GS Scoltand & GS Malvines \\ \hline
	Energy & 21000 & 5000 & 10000 & 10000 \\ \hline
	Maintenance & 8000 & 8000 & 8000 & 8000 \\ \hline
	Internet & 1000 & 1000 & 1000 & 1000 \\ \hline
	\rowcolor[gray]{0.65}
	Total & 30000 & 14000 & 19000 & 19000 \\ \hline
	\bottomrule
\end{tabular}
\caption{Ground Stations' maintenance and operation cost}
\end{center}
\end{table}

And also the costs associated to salaries of the operators:

\begin{table}[h]
\begin{center}
\begin{tabular}{ | l | l | }
\toprule
\hline
	\rowcolor[gray]{0.75}
	Concept & Cost \\ \hline
	Salaries GS Canada & 382000 \\ \hline
	Salaries GS Scotland (UK) & 227000 \\ \hline
	Salaries GS Malvines & 82000 \\ \hline
	Salaries MCC & 430000 \\ \hline
	\rowcolor[gray]{0.65}
	Total & 1121000 \\ \hline
	\bottomrule
\end{tabular}
\caption{Salaries of Ground Stations and Main Control Center operators}
\end{center}
\end{table}

Nevertheless, those costs are operational costs (annual), so they are not to be taken into account when estimating the budget. They will be assumed from the profit of the company. They have been added to this document just as for informative purpose, but won't be taken into account when calculating the budgets. 

\section{Satellites Costs}
The costs of the satellites can be splitted into two big groups: the costs associated to building each satellite and the costs associated to the assembling of them.

Let's start with the building costs for each satellite:

\begin{table}[h]
\begin{center}
\begin{tabular}{ | l | l | l | l | }
\toprule
\hline
\rowcolor[gray]{0.75}
	Component & Units per satellite & Costs (unit) & Cost (satellite) \\ \hline
	Structure & 1 & 3900 & 3900 \\ \hline
	Thermal protection & 1 & 1000 & 1000 \\ \hline
	\rowcolor[gray]{0.85}
	Electric power system &  & \  & \  \\ \hline
	Solar arrays & 4 & 17000 & 68000 \\ \hline
	Batteries & 2 & 6300 & 12600 \\ \hline
	Power management & 1 & 16000 & 16000 \\ \hline
	\rowcolor[gray]{0.85}
	Payload &  & \  & \  \\ \hline
	1st Patch Antenna & 1 & 18000 & 18000 \\ \hline
	Patch antenna & 7 & 7000 & 49000 \\ \hline
	Antenna deployment & 1 & 3000 & 3000 \\ \hline
	Transciever inter-satellite & 3 & 8245 & 24735 \\ \hline
	Transciever space to ground & 1 & 5500 & 5500 \\ \hline
	Data handling system & 1 & 5000 & 5000 \\ \hline
	Variable expenses & 1 & 4000 & 4000 \\ \hline
	\rowcolor[gray]{0.85}
	AOCDS &  & \  & \  \\ \hline
	Thruster & 1 & 50000 & 50000 \\ \hline
	CubeSpace ACDS & 1 & 15000 & 15000 \\ \hline
	\rowcolor[gray]{0.65}
	Total &  & \  & 275735\  \\ \hline
	\bottomrule
\end{tabular}
\caption{Satellite cost (of each satellite)}
\end{center}
\end{table}

As it has been already mentioned, the costs are approximated to its high closes thousand value. Therefore, the cost per satellite is of 276000 \euro. 

Taking into account that there are 189 satellites in Astrea's constellation (21 satellites per plane, and 9 planes), the total building cost is:

\begin{table}[h]
\begin{center}
\begin{tabular}{ | l | l | }
\toprule
\hline
	\rowcolor[gray]{0.75}
	Cost of each satellite & Total cost \\ \hline
	276000 & 52164000 \\ \hline
	\bottomrule
\end{tabular}
\caption{Total satellites cost}
\end{center}
\end{table}


There are also the assembling cost of the satellites:

\begin{table}[h]
\begin{center}
\begin{tabular}{ | l | l | l | }
\toprule
\hline
\rowcolor[gray]{0.75}
	Concept & Cost per unit & Cost per constellation \\ \hline
	Individual Assembling & 20000 & 3780000 \\ \hline
	Constellation Assembling &  & 150000 \  \\ \hline
	\rowcolor[gray]{0.65}
	Total Cost &  & 3930000 \  \\ \hline
	\bottomrule
\end{tabular}
\caption{Satellites assembling cost}
\end{center}
\end{table}

The conclusion is that the global cost of the satellites are:

\begin{table}[h]
\begin{center}
\begin{tabular}{ | l | l | }
\toprule
\hline
\rowcolor[gray]{0.75}
	Concept & Costs  \ \\ \hline
	Building & 52164000 \\ \hline
	Assembling & 3930000 \\ \hline
	\rowcolor[gray]{0.65}
	Total & 56094000 \ \\ \hline
	\bottomrule
\end{tabular}
\caption{Global satellites cost}
\end{center}
\end{table}



\section{Launching Costs}
There are 9 different planes of satellites orbiting the Earth. Each plane requires a different launcher. Therefore, there ir a cost associated to each launcher. Moreover, a fee must be paid too for every satellite carried on in the launcher. 

Consequently, the launching costs are:

\begin{table}[h]
\begin{center}
\begin{tabular}{ | l | l | l | l | }
\toprule
\hline
\rowcolor[gray]{0.75}
	Concept & Individual cost & Number of units & Total cost \\ \hline
	Launcher & 5362000 & 9 & 48258000 \\ \hline
	Satellites in launcher & 16000 & 189 & 3024000 \\ \hline
	\rowcolor[gray]{0.65}
	Total &  &  & 51282000 \\ \hline
\end{tabular}
\caption{Launching cost}
\end{center}
\end{table}

\newpage
\section{Insurance and administration}
There are some other costs which are not taken into account when calculating the budget, similar to the salaries of the Ground Stations and the Main Control Center operators, as has been mentioned. Those costs are the anual insurance cost and the administrarion cost. Those costs are to be paid anually (operation costs), and therefore they are not included in the different budgets. They will be subtracted from the profit of the company. Next it will be shown a summary of them. Again, they are shown just for informative purpose. For further information on those costs, such as their break down, refer to the Annex V. 

\begin{table}[h]
\begin{center}
\begin{tabular}{ | l | l | }
\toprule
\hline
	Administration cost (anually) & 259000  \ \\ \hline
	Insurance cost (anually) & 2246000 \\ \hline
	\rowcolor[gray]{0.65}
	Total & 2505000 \ \\ \hline
	\bottomrule
\end{tabular}
\caption{Administration and insurance anual costs}
\end{center}
\end{table}

\chapter{Budget}
As the introduction of the Budget has already explained, there will be two different budgets. The first one is the initial budget, required at the beginning of the project, and the second one is the periodic budget, required every 5 years from the beginning of the project.

\section{Initial Budget}
This budget is for the beginning of the project (year 0). At that point, the project is designed, the Ground Stations and the Main Control Center is designed, built and started, the satellites of the first cycle of the constellation are built and assembled and eventually launched. Therefore, the costs that must be taken into account are:
\begin{itemize}
\item Manpower (engineering hours): 395000 \euro .
\item Communications Initial Investment: 4068000 \euro .
\item Satellites building and assembling: 56094000 \euro .
\item Launching costs: 51282000 \euro .
\end{itemize}

Adding all those quantities, \textbf{the initial budget turns out to be of 112 M\euro}. 

\newpage
\section{Periodic Budget}
This budget is for every 5 years once the project is started (this is, year 4, 9, 14 and so on). At that point, the only cost is the one derived from the requirement of renewal of the constellation. Therefore, the costs are of building and assembling the satellites and launching them again:

\begin{itemize}
\item Satellites building and assembling: 56094000 \euro .
\item Launching costs: 51282000 \euro .
\end{itemize}

Adding all those quantities, \textbf{the periodic budget (every 5 years) turns out to be of 107 M\euro}. 

This budget could mean another investment required but also could come from the wide benefits that Astrea provides by that time. For further details, refer to the Feasibility Study (section 7 of the Report, or Annex V).



