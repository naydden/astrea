\chapter{Environmental Impact Study}
This chapter pretends to assess the environmental consequences (positive and negative) of developing the project. The target of this study is to identify, predict, evaluate and mitigate the biophysical and social negative effects that the project could generate during the execution of it. The entire study can be seen at the \cite[Chapter 3]{annex5}, since in this document there are only exposed the conclusions of it.
The study is focused in these 3 aspects:

\begin{itemize}
\item \textbf{Ground stations}. It is concluded that they would not represent any environmental impact if they are in a place where they do not interfere with the normal behaviour of any ecosystem.
\item \textbf{Satellites}. The study analyses the impact of the fabrication and the performance. The satellites are certificated by all the corresponding regulations and there is no significant environmental impact during the fabrication. On the other hand the satellites are designed to be burned out in the atmosphere in a period of 5 years, and they would not be orbital waste.
\item \textbf{Launching}. The launching is the most critical part in environmental terms and it is depth analysed. Basing in a Environmental Assessment done by the Ministry for the Environment of New Zealand \cite{EIS} it is concluded that there is no significant damage in the environment at short term.
\end{itemize}

In conclusion the project is environmentally viable.