\subsection{Structure and mechanics}
The design and operation of a CubeSat is a complex process that must be completed keeping in mind the huge differences between all subsystems as well as the role they will play during the lifetime of the mission. And since these systems will operate in space, they have to be prepared and certified to withstand extreme temperature and radiation conditions.

The satellite used by Astrea must have high compatibility between all the systems to avoid potential problems and has to be tested (either all the systems together or one by one). Their correct functioning has to be ensured, especially the critical systems such as the solar arrays, batteries and antennas should be fully operational for at least four years.

\subsubsection{Structure}
The mission of the structure is to sustain and protect all the electronic devices carried by the satellite. In order to ensure that all the electronic and mechanical systems can be mounted upon the structure, a high compatibility between these systems is required; therefore, the structure must be very flexible regarding the arrangement of the subsystems.

The structure chosen is manufactured by \textbf{Innovative Solutions In Space (ISIS)}. Among its features it is worth mentioning that it can withstand the high range of temperature it will face in the space (from -40ºC to 80ºC) and it is highly compatible; almost every physical system  used can be placed within the structure or on its faces (such as the antennas or the deployable solar arrays). Finally, the mass of the structure is relatively low, and given that the mass of the other subsystems is sometimes a drawback, it is plus point. \cite[Chapter 1, Section 1]{annex4}

\subsubsection{Thermal protection}
The CubeSat is vulnerable to suffer extreme temperatures while operating in space, both below zero and above zero. The thermal protection system consists of a set of layers (MLI) made of insulating materials and it aims to protect the CubeSat from potential thermal shocks. The satellite must remain within an optimal range of temperature, despite of the variation of the external temperature, in order to work properly. Furthermore, the thermal protection system should also dissipate the heat produced by the other systems.

\textbf{Dunmore Aerospace} has been chosen to provide us its MLI product. The product, \textbf{Dunmore Aerospace Satkit}, has been designed for small satellites operating in LEO and it will provide the CubeSat with the protection required during operation. \cite[Chapter 1, Section 1]{annex4}

\subsubsection{Options chosen for the structure and thermal protection}
The options chosen are presented in the table \ref{structurefinal}.

\begin{longtable}{| l | r | r | r | }
\hline
\rowcolor[gray]{0.80}	\textbf{System} &  \textbf{Brand and model}     & \textbf{Price per unit (\euro)} & \textbf{N. of units}  \\
\hline
\endfirsthead

	   ~3U Structure & ISIS & 3900 & 1 \\
	   \hline
	   ~Thermal Protection & Dunmore Satkit & 1000 & 1\\
	\hline

\caption{Options chosen for the structure and thermal protection}
\label{structurefinal}
\end{longtable}