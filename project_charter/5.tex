\section{Organization of the group}


\subsection{Hierarchy}
\paragraph{}
Designing a nanosatellite constellation is quite ambitious and requires lots of work because there are many things to consider. In order to build a work strategy, the project is divided in tasks that will be described later on. As the different tasks depend on each other, the project members have decided to follow a hierarchy. Every task is developed by a small team between 2 and 5 people depending on the amount of work the task requires.
\paragraph{}
Each small team has to have a coordinator which has two principal functions. The first one is to manage the group so he is responsible for the good organisation and progression of the task. The second is that he is the voice of the team. That means that the coordinator is the one who represents his work team when transferring information to the other group coordinators and the project managers and vice versa. 
\paragraph{}
Finally over all the teams there is the project manager who maintains order, ensures the project progress and manages people for major decisions. Finally there is also a secretary in charge to write the minutes of each meeting.


\subsection{Documents Organisation}
\paragraph{}
Nowadays, the internet is crucial for teamwork because it provides lots of tools that improve networking such as sharing documents, communicating and even collaborating working. The Astrea team has 17 members so it is essential to define protocol to organise all the documents and information found to take advantage of resources. 
\paragraph{}
The principal communication tool used is \textit{Slack} which is a platform specialised in team communication. \textit{Slack} defines itself as a real-time messaging, achieving and search for modern team which is interesting for us because it allows the group to communicate at all times for punctual doubts and small decisions. For major decisions a date is specified by a \textit{doodle} to meet.
\paragraph{}
Moreover, to share documents we use two platforms: \textit{Slack} and \textit{BSCW}. On \textit{Slack} we put first drafts or documents that can be interesting. \textit{BSCW} is the main information storage because information and documents are stocked and organised in folders.
\paragraph{}
At last, the text editor used to develop the project is Latex which combined with Git allows us to work remotely on a same document without overriding someone else's work. This work system is really interesting for such a big group in order to work on the same document while keeping a record of the changes. 