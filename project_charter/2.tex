\section{Scope of the project}
This section establishes the scope of the project. \\

{\bfseries Satellite development} 
\begin{itemize}
	\item Select the proper satellite’s weight and size, taking into account the next constraints: the launch system cost, the relation between the weight, size and the orbit decay time and, lastly, the interdependency with the selected subsystems.
	\item Deep study of the market and of the state of the art so that later choice on which subsystem to include is done accordingly. The most important subsystems will be analysed. These are: the structural subsystem, the power subsystem, the thermal control subsystem, the attitude control subsystem and the data handling subsystem. The information is going to be extracted mainly online. Also, prestigious magazines can be taken into account as well as contacting some satellite companies.
	\item Eventually, a subsystems choice will be done taking into account the cost, the ease of integration and the need to fulfil the project’s requirements.
\end{itemize}

{\bfseries Orbital design}

\begin{itemize}
	\item The orbit design will be accomplished according to the results of several studies such
as visibility between satellites and between satellites and ground stations. Also, collision
and orbital decay avoidance is going to be taken into account. Finally, stated requirements as low latency or the possibility to act in case of a network’s failure are going to be contemplated due to their tight dependency on the selected orbit.
	\item The number of satellites and the number of orbital planes will be deducted from those
studies.
	\item A study will be carried out to clarify if the Earth is the only celestial body that will
influence the satellites or others, for instance, the Moon or the Sun will also have to
be considered. It will consist in the inclusion of empirical or physical models in the orbit calculation software and evaluate the level of significance of these cellestial bodies in the results. 
	\item The specific existing legislation will be taken into  account and followed during all
the orbit development.

\end{itemize}

{\bfseries Constellation Deployment} 

\begin{itemize}
	\item A comparison among the existing launch platforms will be carried out to find out the
one that fulfils the mission requirements with a reasonable economic conditions.
	\item A launching date will be reserved if the chosen launch platform requires it.
	\item The recommendations of \textit{Joint Space Operation Center} will be followed and their application form will be followed up to ensure all the launch procedure accomplishes the legislation.
 	\item An end of life strategy will be designed according to the CubeSats lifespan, orbit decay,
replacement stratagem of the company and legislation procedures.
\end{itemize}

{\bfseries Operation} 

\begin{itemize}
	\item An analysis will be done to clarify how many ground stations must operate and the possibility of placing a central one in UPC ESEIAAT.
	\item The requirements and costs of the ground station will be determined. 
	\item Communication logistics will be defined.
	\item Communication logistics will be defined. Thus, how the satellites decide whether to send the data or to store it, and if they are to send, where they should do it, is going to be approached. In other words, a high level communications protocol is going to be defined. 
\end{itemize}

{\bfseries Exhibition}

\begin{itemize}
	\item It will consist on a simulation of the constellation. Basically, the results from the orbit’s calculations are going to be used here in order to show the client the finish state of the product. A CAD of the Satellite node is going to be used as well.

\end{itemize}