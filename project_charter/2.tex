\section{Scope of the project}
This section establishes the scope of the project. \\\\
	First of all, a preliminary design will be carried out in order to establish the basic parameters of the constellation and de satellites and to set a common point on which start de following areas of the project. This preliminary design will be based on the existing constellations in the market, the main requirements showed in the next section of this document and a few calculations. \\
{\bfseries Satellite development} 
\begin{itemize}
	\item An establishment of the mission requirements that influence any parameter of the satellite will be developed. These requirements include size, weight, spacecraft subsystems and payload. 
	\item The design of the satellites' payload will be carried out. 
	\item The spacecraft subsystems (structure, power systems, thermal control, telemetry and command and AOCS) will be chosen among the offer of the market, following the stated requirements and ensuring the compatibility with the designed payload. The exact  disposition of the different elements inside de CubeSat won't be indicated. 
 
\end{itemize}

{\bfseries Orbital design.}

\begin{itemize}
	\item The orbit design will be accomplished according to the results of several studies such as visibility between satellites and between satellites and ground stations, collision avoidance, orbital decay avoidance and stated requirements as global coverage, low earth orbit, low latency.  These studies won't be carried out to the last detail. The two main priorities of this reserach will be the global coverage keeping the costs at a reasonable level. 
	\item The number of satellites and the number of orbital planes will be deducted from those studies. 
	\item A study will be carried out to clarify if the Earth is the only celestial body that will influence the satellites or others, for instance, the Moon or the Sun will also have to be considered. This study will consist on a research of information and several calculations in order to arrive to a reliable conclusion.  
	\item The specific existing legislation will be taken into an account and followed during all the orbit development.
\end{itemize}

{\bfseries Constellation Deployment.} 

\begin{itemize}
	\item A comparison among the existing launch platforms will be carried out to find out the one that fulfills the mission requirements and a reasonable economical conditions.
	\item If the launch platform requires a window to be booked, a date will be chosen and the procedure that need to be followed to book it will be clarified.
	\item The recommendations of Joint Space Operation Center will be followed and their application form will be filled up to ensure all the launch procedure accomplishes the legislation. 
\end{itemize}

{\bfseries Operation.} 

\begin{itemize}
	\item An analysis will be done to clarify how many ground stations must operate and the possibility of placing a central one in UPC ESEIAAT.
	\item The requirements and costs of the ground station will be determined. 
	\item Communication logistics and protocols will be defined.
	\item An end of life strategy will be designed according to CubeSat lifespan, orbit decay, replacement stratagem of the company and legislation procedures. 
\end{itemize}

{\bfseries Exhibition.}

\begin{itemize}
	\item It will consist on a simulation of the constellation to show to the client how it works. Specifically, it will be displayed in video format.

\end{itemize}